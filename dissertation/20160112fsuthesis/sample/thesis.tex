%%% This is a sample thesis showing how to use the FSU Thesis Class.
%%% This file is the main LaTeX source file.  It contains all the
%%% front matter and back matter of the document.  The main text
%%% (chapters and appendices) are '\input'd from this file.  Lines
%%% that begin with a '%' are "comments", and they are completely
%%% ignored by LaTeX.

%%% The document begins by declaring the document class as 'fsuthesis'
%%% to be set in 11pt type.  I also turn on the copyright notice, for
%%% testing purposes.  The 'expanded' option enables wider baseline
%%% spacing throughout the document, except where it needs to be
%%% single-spaced.  This is now required by FSU.

\documentclass[11pt,copyright,expanded]{fsuthesis}

%%% You can turn off "double-spacing" by eliminating the 'expanded'
%%% option.  This might be handy if you want a working version of
%%% your document that consumes less paper.  You can use this instead:
%%%
%%%   \documentclass[11pt,copyright]{fsuthesis}
%%%
%%% See the User Guide and the references for additional document
%%% options that may be useful to you.

%%% Many hundreds of additional packages extend the basic LaTeX
%%% features and functionality.  They are included by the command
%%% '\usepackage{packagename}'.  "The LaTeX Companion" is a great
%%% resource for delving \usepackage{natbib}into some advanced formatting and
%%% functionality among LaTeX's many packages.

%%% Be aware that some packages may break some of the formatting
%%% features provided by by 'fsuthesis' class. In particular, DO
%%% NOT USE 'geometry' or 'setspace' packages with 'fsuthesis':
%%% these will mess up the spacing badly.

%%% I'll be using some of the math formatting options from the
%%% American Mathematical Society math package, so I include a line
%%% for that here.
\usepackage{natbib}
\bibliographystyle{abbrvnat}

\usepackage{amsmath}

%%% The mflogo package provides the METAFONT logo characters. Most
%%% people won't need this (unless you're writing about METAFONT).

\usepackage{mflogo}         % (not needed for general use)

%%% The textcase package is useful if I have math formulas or symbols
%%% in any chapter or section headings, preventing lower-case math
%%% symbols from being upper-cased.

\usepackage[overload]{textcase}

%%% The default citation and bibliography styling provided by LaTeX
%%% may not be the standard in your discipline.  The 'natbib' package
%%% extends the formatting options for citations and bibliographic
%%% references common in the sciences, and many disciplines define
%%% their own style guides for bibliographies and citations through
%%% 'natbib'.  If you choose to use 'natbib' here, you'll also need to
%%% select the proper 'natbib' style file in the bibliography portion
%%% of this file later on.  See the 'natbib' package documentation for
%%% more information.

%\usepackage[round]{natbib}       % an alternative bibliography package

%%% Yet another bibliography alternative for people who need APA-style
%%% citations and bibliography formatting is the 'apacite' package.
%%% Most people will want to use the 'natbibapa' compatibility option.
%%% (You must have the 2012 or later version of the 'apacite' package
%%% to use this option. If you use the hyperref package, you *must*
%%% use the 'natbibapa' option, or there will be citation errors.) See
%%% the 'apacite' manual for more detailed information on using this
%%% package and formatting your bibliography.

%\usepackage[natbibapa]{apacite}  % an alternative bibliography package

%%% The following few lines are a little different: if I want to
%%% typeset my thesis using 'pdflatex', I need to use the 'pdftex'
%%% driver to insert figures, and I'll also turn on hyperlinks in my
%%% document.  If instead I'll be producing a DVI or PostScript
%%% document (which I can convert to PDF later), I will want to use
%%% the 'dvips' driver for the graphicx package, and I won't turn on
%%% hyperlinks.  I can accomplish this by testing whether or not I'm
%%% running in PDF mode using the '\ifpdf' test below.  If I'm in PDF
%%% mode, then all my figures will need to be either PDF files, PNG
%%% files, or JPEG files.  If I'm using the 'dvips' driver, then any
%%% figures I include should be EPS (Encapsulated PostScript).  (This
%%% sample document provides figure files in both formats for testing
%%% purposes, so I can select either driver here.)  These days, most
%%% people will probably want to use 'pdflatex' to create a PDF file
%%% directly, so this extra logic is not essential; you may just enter
%%% '\usepackage[pdftex]{graphicx}' in your document directly.

\ifpdf   % We execute this part (up to \else) if we are in PDF mode.
  \usepackage[pdftex]{graphicx}
  % We can create color text in our document using the color package.
  % We load it here so that we can tweak the default colors used in
  % the links created by hyperref, as the defaults are a bit too
  % bright and gaudy for an austere document such as this. ;-)
  % Note that the Manuscript Clearance Advisors will not allow any
  % colored text at all, so if you're using hyperref, you'll need to
  % set all the colors to black (or set colorlinks=false below).
  \usepackage{color}
  \definecolor{mygreen}{rgb}{0,0.6,0}
  \definecolor{myblue}{rgb}{0.3,0.2,0.8}
  \definecolor{myred}{rgb}{0.8,0.1,0.1}
  % The hyperref package has lots of features and options.  If you
  % are interested in creating a state-of-the-art PDF document out
  % of your thesis/dissertation, you should read more about this
  % extremely useful package.
  \usepackage[colorlinks=true,bookmarks=true,pdfborder={0 0 0},
    linkcolor=myblue,urlcolor=myred,citecolor=mygreen,
    breaklinks=true,bookmarksnumbered=true]{hyperref}
\else    % Otherwise we'll run this part if we are not in PDF mode.
  \usepackage[dvips]{graphicx}
\fi

%%% The following lines increase LaTeX's default penalties for club
%%% lines and widow lines, which will discourage creating clubs and
%%% widows in all but the most extreme cases.  (If the last line on a
%%% page is the first line of a new paragraph, it's called club line.
%%% If the first line of a page is the last line of a paragraph, it
%%% is called a widow line.  Either of these conditions look a little
%%% ugly, typographically speaking.)

\widowpenalty=9999
\clubpenalty=9999

%%% Normally, the fsuthesis class will allow the bottom line of the
%%% text column on each page to float up and down as necessary to keep
%%% the vertical spacing on the page consistent.  You can disable this
%%% behavior and force the baseline of every page to be stretched to
%%% fill the allotted page boundaries by enabling \flushbottom.  (It's
%%% commented out here, so the default behavior 'raggedbottom' is in
%%% effect.)

%\flushbottom
%\raggedbottom  % This is the default.  However, if you have specified
                % the 'twoside' document option, you'll need to turn on
                % '\raggedbottom' to keep the same behavior, since
                % '\flushbottom' becomes the the default with 'twoside'.

%%% By default, section headings down to the \subsection level will be
%%% entered into the Table of Contents.  If that is too much detail, I
%%% can limit the entries to just the \section-level headings (for
%%% example) by resetting the 'tocdepth' parameter to '1'. (The User
%%% Guide explains this in more detail.)

%\setcounter{tocdepth}{1}

%%% Next, we set up some of the document description elements for use
%%% on the title and committee pages.  If the college or school does
%%% not sub-divide into departments for thesis/dissertation
%%% publication purposes, then you should delete the '\department{}'
%%% macro below.  Note that we use standard title capitalization
%%% conventions here.

\title{Of Cabbages and Kings:\protect\\
  An Analysis of the Use of the Colon in Dissertation Titles}
                                      %% Manuscript title
\author{V\`\i ct\"or \'E. I\c smyne}  %% Testing accented characters in name
\college{College of Arty Science}     %% School or College
\department{Department of Furtive Studies}  %% Delete if no department
\manuscripttype{Thesis}               %% [Thesis, Dissertation, Treatise]
\degree{Master of Science}            %% Name of the degree
\degreeyear{2015}                     %% Graduation Year
\defensedate{May 5, 2015}             %% Date of Defense

%%% The following options insert additional metadata into a PDF file if
%%% you are using 'pdflatex' to process your document.  This information
%%% is not printed in the paper, but the information will be available
%%% in the document properties tab of Acrobat Reader (for example).
%%% The document title and the author name are already inserted
%%% automatically.  Metadata can help search engines locate your
%%% document in searches.  The "subject" should be a concise
%%% description of the paper's topic. The "keywords" should be
%%% relevant terms that would help other researchers in your field
%%% locate your paper.  Keywords should be separated by semicolons.

\subject{Dissertation Formatting}     %% PDF document metadata subject
\keywords{latex; fsuthesis; etd; tables; figures; bibtex; document formatting}
				      %% PDF document metadata search
                                      %%   keywords, separated by semicolons

%%% Assuming I have five professors on my committee, I will need five
%%% lines here, one for each committee member.  The professors should
%%% be specified in the order in which they should appear on the
%%% committee page.  The first argument of the \committeeperson macro
%%% is the professor's name (without title), and the second argument
%%% is the professor's committee membership position.

\committeeperson{Faux Causson Yorverk}{Professor Directing Thesis}
\committeeperson{Verda Boizaar}{University Representative}
\committeeperson{Beauxeau D'Claune}{Committee Member}
\committeeperson{Arlip Zarseeld}{Committee Member}
\committeeperson{Tumen Enoats}{Committee Member}
%committeeperson{Deb O'Nair}{Committee Member}
\committeeperson{Auss M'Pauers}{Committee Member}

%%% Here I am defining some additional LaTeX macros for use throughout
%%% my document.  These are intended to simplify some typing and to
%%% encourage uniformity, since I won't always have to remember, for
%%% example, that acronyms should be in a smaller ALLCAPS font: I can
%%% just type '\acro{text}'.  Note that though I am defining these
%%% macros here, they will be accessible to any other source text
%%% which follows, even if the source text is in a completely
%%% different file that I'll be \input-ing later. You are free to
%%% define your own macros in a similar fashion. You'll probably want
%%% to read one of the references for more detailed information about
%%% creating your own macros.

\newcommand*{\acro}[1]{{\small\textsc{#1}}}
\newcommand*{\lit}[1]{\texttt{#1}}
\newcommand*{\pkg}[1]{\textsf{#1}}

%%% The initial document setup is done.  Now we tell LaTeX that our
%%% text will begin.

\begin{document}

\frontmatter          %% Establish small roman numeral numbering

%%% The following two lines use the information we have already
%%% provided to create the title page and the committee page
%%% automatically, so you don't need to do anything fancy.

\maketitle            %% Create the title page
\makecommitteepage    %% Create the committee page

%%% If I wish to have a dedication, then I will use the dedication
%%% environment.  This is optional, so if you don't need a dedication
%%% page, you may delete everything between (and including)
%%% the \begin{dedication} and \end{dedication} text.  The format of
%%% the page is entirely up to you.  The dedication page creates no
%%% heading.

\begin{dedication}
\centering
To my parents, who always suspected I'd end up here
\end{dedication}

%%% Likewise, I use the acknowledgments environment to create an
%%% acknowledgments page.  Delete this section if there will be no
%%% acknowledgments page.  The heading is inserted automatically.

\begin{acknowledgments}
Many thanks are due to many people.  My major professor didn't
know what she was getting herself into when she took me on as
a student, and I will always be grateful for her support and
guidance.  The other members of my committee deserve hazard pay,
and this paper would not be the same without their diligence:
many thanks.
\end{acknowledgments}

%%% I will need all the contents here except the list of musical
%%% examples.  (These pages will be generated automatically by LaTeX
%%% as it encounters captions and table and figure environments later
%%% in the document.)  I only need to specify the figure and table
%%% lists here if I have more than one figure or table,
%%% respectively. My sample document does have just one musical
%%% example, so I don't need to turn on that option.

\tableofcontents
\listoftables
\listoffigures
%\listofmusex

%%% I would like to present a short list of symbols.  By using the
%%% listofsymbols environment, the proper entry will be made in the
%%% Table of Contents.  Other than the heading, the rest of the page
%%% is entirely up to me.  I will use a LaTeX tabular environment to
%%% list the data.

\begin{listofsymbols}
The following short list of symbols are used throughout the document.
The symbols represent quantities that I tried to use consistently.
\begin{center}
\begin{tabular}{ll}
$\pi$&$3.1415926\ldots$\\
$E$&$mc^2$\\
$F$&$ma$\\
$R_e$&Mean Radius of the Earth${}\approx 6367.65\,\textup{km}$\\
$e$&Base of Natural Logarithms${}\approx 2.71828\ldots$\\
$P$&The principal borrowed\\
$N$&The number of payments\\
$i$&The fractional (periodic) interest rate\\
$P_j$&The principal part of payment $j$\\
$I_j$&The interest part of payment $j$\\
$B$&A final balloon payment\\
$x$&The regular payment\\
$R$&The principal remaining after $r$ payments \\
$r$&Some number of payments such that $0 < r < N$ \\
$R_j$&The principal remaining after $j$ payments \\
$A_j$&The total interest paid out after $j$ payments
\end{tabular}
\end{center}
\end{listofsymbols}

%%% You may also create a list of abbreviations if it may be of use to
%%% your readers.  Otherwise, this section may be deleted or remain
%%% commented.  Instead, if you need another frontmatter environment
%%% for some other use (besides a list of abbreviations or symbols),
%%% you could rename the heading to suit your purposes.  For example,
%%% if I need an Index of Scary Movies, I can steal the
%%% 'listofabbrevs' environment and rename its heading.  (The
%%% environment name stays the same, but heading will be changed on
%%% the page and in the table of contents. The 'listofsymbols'
%%% environment above may also be co-opted in this way.) Here's how I
%%% might do this:

%\renewcommand*{\listabbrevname}{Index of Scary Movies}
%\begin{listofabbrevs}
%  [... insert scary movie index material here ...]
%\end{listofabbrevs} 

%%% Now I use the abstract environment to create an abstract page.
%%% The heading is created automatically.  The rest of the content
%%% is up to you.
\begin{abstract}
The FSU Thesis Class is a \LaTeX\ document class useful for writing
Theses, Dissertations, and Treatises.  It has several custom macros
and environments which are intended to ease the burden of formatting
for writers of these documents so that they may focus more on the
research and presentation rather than on the page layout. This sample
document is intended to provide a few examples of how most of the
class features may be used.

The main source file for this document is \lit{thesis.tex}, and this
is where you should start reading.  The document's source is spread
over several files.  Many of the files contain helpful
\LaTeX\ comments which are not printed out here.  It may be
instructive to look at the source files as you read this ``output'' to
see how the document was created.
\end{abstract}

%%% The abstract is the last element of the so-called "front matter"
%%% of the document.  We now move on to the "main matter" --- the
%%% chapter material (and optional appendices).  By calling the
%%% \mainmatter macro, we reset page numbering to arabic numerals
%%% beginning with '1'.

\mainmatter

%%% I have chosen to create one new LaTeX file for each chapter of my
%%% document.  This decision is arbitrary.  I may break the document
%%% into as many or as few pieces as I like.  I may name the chapter
%%% and appendix files anything I like (though I might have some
%%% trouble if I use spaces or odd characters in my file names).  The
%%% only requirement is that the file name ends with the extension
%%% '.tex'.  For the \input lines below, I can leave out the '.tex'
%%% extension.  If instead I prefer to create one long file containing
%%% my entire thesis, I may just continue on here by starting with the
%%% first \chapter command of my document.  (In this case, I would not
%%% need the following '\input' commands at all.)  I have chosen to
%%% name my chapter files by the concepts they contain.

\input chapter1 
\input chapter2
\input chapter3
\input chapter4

%%% Having finished the main text of my document, I will move on to
%%% the appendices (or my one appendix, in this case).  An appendix
%%% will look exactly like a chapter, even including a \chapter macro
%%% call.  By first calling the \appendix macro, I will cause all
%%% subsequent chapter headings to be labeled "Appendix" and to be
%%% enumerated by letters rather than numbers.  I.e., the first
%%% appendix will be titled 'Appendix A', then 'Appendix B', and so
%%% on.  This also adds a heading entry to the Table of Contents
%%% labeled 'Appendix'.  If you have more than one appendix, you may
%%% want to change the ToC heading to read 'Appendices'.  If you need
%%% to do this, you may reset the value of '\appendixtocname' before
%%% the call to the \appendix macro:

%\renewcommand*{\appendixtocname}{Appendices} % For more than one appendix
\appendix
\input appendix

%%% I use the following chapter to test how things look when printed
%%% in 'expanded' spacing and to try out other LaTeX and FSUthesis
%%% class features not presented previously.  It's not included in the
%%% sample document by default, but you may uncomment it if you like.
%%% Including this file also causes the Table of Contents to exceed a
%%% single page in size, so it's helpful for testing those aspects.

%\input tests
%\input history  % just for fun

%%% At this point, we are moving into the document's "back matter".
%%% The first element is the references/bibliography section.
%%% According to the FSU "Guidelines & Requirements for Thesis,
%%% Treatise, and Dissertation Writers", this section should be called
%%% "References" if it contains only source material cited in the
%%% document; and it should be called "Bibliography" if it contains a
%%% broader scope of material than that actually cited.  Because this
%%% document only lists cited materials, I will need to be sure the
%%% heading is labeled "References".  The FSU Thesis Class "User
%%% Guide" tells me to use \renewcommand to accomplish this:

\renewcommand*{\bibname}{References}

%%% The skeleton document in the thesis-template folder gives an
%%% example of the "simple" 'references' environment.  In that
%%% document, you must format all of your own references and manage
%%% your own citation styling.  In contrast, this document assumes
%%% that I will be using BibTeX to format my references.  I'll be
%%% using the LaTeX 'plain' format style, and the bibliographic data
%%% are kept in the file 'myrefs.bib'.  (I leave off the file
%%% extension in the \bibliography command below.)

%%% Plain LaTeX provides just a few citation and bibliography styles.
%%% Many more styles are available using 'natbib', 'apacite', or other
%%% packages. A quick web search with 'bibtex' and your own discipline
%%% as keywords may help you locate both references and the
%%% appropriate style files for you to download. If you
%%% '\usepackage{natbib}' or '\usepackage{apacite}' at the top of this
%%% file, then you should comment out the 'plain' style below, and
%%% uncomment the appropriate alternative style for the package you've
%%% chosen.

%\bibliographystyle{plain}     % This is the default LaTeX/BibTeX style
%\bibliographystyle{unsrtnat} % If using 'natbib', use this line instead
%\bibliographystyle{apacite}  % If using 'apacite', use this line instead

%%% The following line actually generates the bibliography using the
%%% data supplied in 'myrefs.bib', the citations in my document, and
%%% the style I've selected above.

\bibliography{myrefs}

%%% The last element of the document is the biographical sketch.  The
%%% heading and table of contents entry are automatically created by
%%% using the biosketch environment.  The rest of the content is up to
%%% you.  Remember not to include any personal contact info.

\begin{biosketch}
The author was born, and then the author was ``educated,'' at least to
some degree.  After finishing high school in Florida, the author
completed a Bachelor of Arts degree at Florida State University.
Following a decade in the work force in his discipline, the author
returned to FSU to pursue graduate work.
\end{biosketch}

%%% Here endeth the document.  LaTeX will ignore anything that follows
%%% the \end{document} command.

\end{document}
