%\documentclass[11pt,letterpaper]{article}
\documentclass[11pt,letterpaper]{ltxdockit}[2011/03/25]
\usepackage[american]{babel}
\usepackage{shortvrb}
\usepackage{pifont}
\usepackage{libertine}
\usepackage[scaled=0.8]{beramono}
\usepackage{microtype}
\usepackage{makeidx}

\lstset{basicstyle=\ttfamily,keepspaces=true}
\KOMAoptions{numbers=noenddot}
\definecolor{spot}{rgb}{0.5,0.1,0.1}
\definecolor{boxfill}{rgb}{0.99,0.98,0.95}

\addtokomafont{paragraph}{\spotcolor}
\addtokomafont{section}{\spotcolor}
\addtokomafont{subsection}{\spotcolor}
\addtokomafont{subsubsection}{\spotcolor}
\addtokomafont{descriptionlabel}{\spotcolor}

\setlength{\paperwidth}{8.5in}
\setlength{\paperheight}{11in}
\setlength{\pdfpagewidth}{\paperwidth}
\setlength{\pdfpageheight}{\paperheight}
\addtolength{\textheight}{\baselineskip}

\newcommand*{\acro}[1]{{\small\textsc{#1}}}
\newcommand*{\booktitle}[1]{\textit{#1}}
\newcommand*{\complit}[1]{\texttt{#1}}
\newcommand*{\pkg}[1]{\textsf{#1}}
\newcommand*{\comm}[1]{\texttt{\textbackslash#1}}

\newcommand*{\fsuth}{\pkg{fsuthesis}}

\renewcommand{\-}{\discretionary{}{}{}}

% Logos:
\def\BibTeX{{\rmfamily B\kern-.05em%
   \textsc{i\kern-.025em b}\kern-.08em\TeX}}
\def\MiKTeX{MiK\TeX}
\def\MacTeX{\textsf{Mac}\TeX}
\def\TeXLive{\TeX\,Live}

\newenvironment*{macrolist}
  {\list{}{%
      \setlength{\labelwidth}{1.5in}
      \setlength{\labelsep}{10pt}
      \setlength{\leftmargin}{0pt}
      \setlength{\parsep}{0pt}
      \setlength{\listparindent}{\parindent}
      \renewcommand*{\makelabel}[1]{{\setbox0=\hbox{\parbox[t]{1.5in}{\raggedleft\leavevmode\marglistfont##1}}\dp0=0pt\box0}}}}
  {\endlist}

\newenvironment*{narrowref}%
 {\par\frenchspacing
  \setlength{\leftskip}{4em}
  \setlength{\rightskip}{3em}
  \setlength{\parindent}{-2em}
  \setlength{\parskip}{0.6\baselineskip}\ignorespaces}
 {\par\addvspace{.5\baselineskip}}

\DeclareListParser*{\listofitems}{,}

\def\cmditem#1{\cmd{#1}\\}
\def\plainitem#1{#1\\}

\def\macroitems#1{\item[\listofitems{\cmditem}{#1}]}
\def\envitems#1{\item[\listofitems{\plainitem}{Environment\rlap{:},#1}]}
\def\optionitems#1{\item[\listofitems{\plainitem}{#1}]}

\setlength{\marginparsep}{10pt}
\setlength{\marginparwidth}{1.5in}

\def\margcmds#1{\leavevmode\marginpar{\raggedleft\leavevmode\marglistfont\listofitems{\cmditem}{#1}}\ignorespaces}
\reversemarginpar
\def\margenvs#1{\leavevmode\marginpar{\raggedleft\leavevmode\marglistfont\listofitems{\plainitem}{Environment\rlap{:},#1}}\ignorespaces}
\reversemarginpar
\def\margplain#1{\leavevmode\marginpar{\raggedleft\leavevmode\marglistfont\listofitems{\plainitem}{#1}}\ignorespaces}

\setcounter{tocdepth}{2}
\raggedbottom

\title{The \textsf{fsuthesis} \LaTeX{} Class:\\
  A User's Guide}
\author{Bret D. Whissel\\Information Technology Services\\Florida State University}
\date{January 12, 2016}

\makeindex

\begin{document}

\maketitle

\tableofcontents

\section{Introduction to \LaTeX}
\label{sec:intro}

If you are already a \TeX/\LaTeX{} convert, you may skip over this
introductory material and jump ahead to the description of the
\fsuth{} class macros in section~\ref{sec:class}.  If you're new to
\LaTeX, you may want to learn a little bit more about what you may be
getting yourself into~first.

If you have grown up only learning to use the word-processing tools
that are installed on a typical PC, \LaTeX{} may feel awkward
initially.  However, \LaTeX's ability to generate cross-references,
lists of tables and figures, and a table of
contents---automatically---is already worth the small amount of effort
required to get started with this very powerful typesetting system.
Further, if your document contains mathematics, you'll be hard-pressed
to find better software for making your equations look good in~type.

Historically, \LaTeX{} is not a \acro{WYSIWYG}\footnote{pronounced
  ``wizzywig'': What You See Is What You Get}\index{WYSIWYG} system.  Instead,
documents are created using any available plain text editor.  Your
document will contain ``markup'' commands that identify things like
chapter titles, section headings, citations, quotations, enumerated
lists, etc.  When ready, the document is run through \LaTeX{} to
produce viewable or printable output.  This two-step process may be
different from what you're used to, but one advantage is that it
allows authors to focus more on the content of their documents, and to
focus less on the formatting (or at least to defer the attention to
formatting until the final stages of document preparation).

\section{Installation}%
\index{latex@\LaTeX!installation}%
\label{sec:install}

\subsection{The Basic \LaTeX{} System}

The \LaTeX{} system (and the \TeX{} engine upon which it is built)
will take just a little effort and download time to get installed and
running.  It is completely free software with a large and committed
group of users, and there are lots of resources for helping you to get
started.  These are much more comprehensive than this
\booktitle{User's Guide} can~be.

If you are working in a Microsoft Windows environment, take a look at
the \MiKTeX\ project\index{latex@\LaTeX!distributions!\MiKTeX} (see
\url{http://miktex.org}).  Mac users will find the
\MacTeX\index{latex@\LaTeX!distributions!\MacTeX} resources useful (see
\url{http://www.tug.org/mactex}).  Linux/\acro{UNIX} users should
investigate the \TeXLive\index{latex@\LaTeX!distributions!\TeXLive}, if
\TeX{} is not already a part of your installation (see
\url{http://www.tug.org/texlive}).

\subsection{Editors and IDEs}

In addition to the \TeX/\LaTeX{} system, you will need a plain text
editor.  Both of the recommended packages for Windows and Mac provide
an editor suitable for editing your document. Linux/\acro{UNIX}
enthusiasts will probably already have access to and familiarity with
\complit{emacs} or \complit{vim} (both of which have also been ported
to Windows and Mac environments).

There are more advanced document editors for \LaTeX\ which include
document previews or \acro{WYSIWYG} functionality. One such option for
Windows users is \textsf{\TeX nicCenter}\index{latex@\LaTeX!editors!\TeX
  nicCenter}, available from \url{http://texniccenter.org/}.  This
application provides a complete editing and previewing environment for
creating \LaTeX\ documents, including drop-down menus for common
\LaTeX\ commands.  The application
\textsf{Texmaker}\index{latex@\LaTeX!editors!Texmaker} (see
\url{http://www.xm1math.net/texmaker}) provides an integrated
development environment for all three major operating system
platforms.

Another popular option is the
\textsf{LyX}\index{latex@\LaTeX!editors!LyX} document processor (see
\url{http://www.lyx.org}).  However, in order to use the features
provided by the \fsuth\ class, you will need to create a custom layout
for \textsf{LyX} that corresponds to \fsuth.  Unfortunately, this is
not trivial.  If you wish to use \textsf{LyX}, you could try creating
your document using the standard \LaTeX\ \pkg{report} class layout
within \textsf{LyX}, export the document to \LaTeX, and then do the
final editing using the standard tools.

\subsection{Installing the \fsuth{} Class File}

The \fsuth\ class\index{fsuthesis@\pkg{fsuthesis}!package!contents} is
distributed as a zip file.  When the zip file is unpacked, a folder
called \complit{fsuthesis} is created.  Inside that folder will be
found this \booktitle{User's Guide} (in both its \acro{PDF} and
\LaTeX{} source form), the \complit{thesis-template} folder, a
\complit{sample} folder, and supporting files.  The \fsuth{} class
file is called \complit{fsuthesis.cls}.

You may use the folder \complit{thesis-template} as a starting point
for your document: the \complit{fsuthesis.cls} file is already
unpacked there and ready to go. You can copy the template folder to a
new location and build your document within as a self-contained
entity. No further installation is necessary.

If you are already a \LaTeX\ enthusiast, you may make customizations
to the \fsuth\ class yourself.  To do this, you will probably want to
take a look at the documented version of the \fsuth\ class definitions
found in the \complit{fsuthesis.dtx}
source\index{fsuthesis@\pkg{fsuthesis}!source code} file in the main
directory.  Processing this file with \LaTeX, you'll create a
nicely-formatted and indexed version to read:
\begin{verbatim}
pdflatex fsuthesis.dtx 
makeindex -s gglo.ist -o fsuthesis.gls fsuthesis.glo
makeindex -s gind.ist -o fsuthesis.ind fsuthesis.idx
pdflatex fsuthesis.dtx 
\end{verbatim}
Be sure to document your changes, and only edit the
\complit{fsuthesis.dtx} file. If you make changes to the
\complit{fsuthesis.cls} file, they may be overwritten later if the
class file is re-extracted.

To extract a new class file from the altered source, run the
following command:
\begin{verbatim}
latex fsuthesis.ins
\end{verbatim}
To make the class available
system-wide\index{fsuthesis@\pkg{fsuthesis}!installation}, copy the
\complit{fsuthesis.cls} file into the \LaTeX{} file search tree.  (The
proper location is operating system and installation-dependent.  For
\acro{UNIX}/Linux systems, this location might be something like
\complit{/usr/\-share/\-texmf-site/\-tex/\-latex/\-fsuthesis/}.)
Otherwise, for local use only, copy the new \complit{fsuthesis.cls}
file to the folder where your thesis or dissertation document resides.

\section{Helpful \LaTeX\ References}
\label{sec:references}

For simple texts, you might not need more from
\LaTeX\index{latex@\LaTeX!references} than what's described in this
\booktitle{User's Guide}.  For more complicated texts, however, or for
documents containing several tables, figures, or mathematics, you will
certainly want to supplement your \LaTeX{} references.  You will find
a wealth of information on-line using your favorite web search engine,
as well as several bound and printed reference materials.  I have
found the texts cited below to be of particular value.
\begin{itemize}
\item
For first-timers,
\href{http://www.ctan.org/tex-archive/info/lshort/english/lshort-letter.pdf}%
{\booktitle{The Not So Short Introduction to \LaTeXe}}
by Tobias Oetiker, Hubert Partl, Irene Hyna, and Elisabeth Schlegl
promises to have you off and running in a few hours' time.  It's a
document you may find readily on-line in \acro{PDF} form.

\item
The standard reference is the book \booktitle{\LaTeX: A Document
  Preparation System}, 2nd~Ed., by Leslie Lamport, the original author
of \LaTeX{}.  This text covers all the basics clearly and succinctly.

\item
A larger starting reference book is \booktitle{Guide to \LaTeX},
4th~Ed., by Helmut Kopka and Patrick W.~Daly.  At twice the length of
the Lamport book, \booktitle{Guide} covers all the basics, and it also
touches on a few of the more common add-on packages.  The book
comes with a CD-ROM with the \TeXLive\ distribution included, which
can save you a lot of downloading time.

\item
Once your working knowledge of \LaTeX{} is secure, \booktitle{The
  \LaTeX{} Companion}, 2nd~Ed., by Frank Mittelbach and Michel
Goossens covers a broad range of topics and \LaTeX{} add-on packages.
This text goes far beyond the basics, but it's an indispensable
reference if you're interested in customizing the appearance of
\LaTeX{} documents.
\end{itemize}

\section{Working with \LaTeX}

Files you create for processing by \LaTeX{} should have file
extensions of \complit{.tex}, for example, \complit{mythesis.tex}.
For your own convenience, you may split the document into pieces
(perhaps one file per chapter), which may make the editing process a
little easier by keeping manageable the amount of text you must scroll
through at any one time.

While you're typing your document, you will insert macro commands that
mark up elements of your document, indicating chapter and section
headings, equations, tables, figures, etc.  Markup languages attempt
to separate the content of the document from its appearance.  As an
author, you need not be quite as concerned about how everything looks,
just what it says.  By marking up your document appropriately, you can
let \LaTeX{} worry about how everything looks.

\subsection{Paragraphs and Space}
Typing your manuscript for processing by \LaTeX\ is not hard, but
there are a few rules you should know. To end a
paragraph\index{paragraph} and begin a new one, simply leave a
blank\index{blank line} line between them.  No matter how many blank
lines you leave between paragraphs, it's the same as typing a single
blank line: all the extra space is ignored.  Likewise, a 100~taps of
the space\index{space} bar has the same result as a single space.

Don't worry about where the end-of-line occurs as you're typing:
\LaTeX\ will reformat the paragraph for you.  \LaTeX\ will also insert
paragraph indentations, so there's no need to type tabs or spaces (and
they'll be ignored anyway).

The \fsuth\ class has defined most of the document spacing for you, so
you generally don't need to worry about it.  If you have particular
spacing needs, look into the references in
section~\ref{sec:references} for more information.  Here's an example
of how \LaTeX\ processes what you type into paragraphs:

\medskip\noindent\hskip3em
\begin{minipage}[t]{2.1in}\rightskip=0pt plus1.5in
\noindent\textit{Typed as \ldots}\\[5pt]
\noindent\small\texttt{I am typing a silly\\
paragraph to see how it
will turn out.\\
\\
And here is the
next paragraph.}\par
\end{minipage}
\hfill
\begin{minipage}[t]{2.4in}
\parindent=1em\rightskip=3em
\noindent\textit{Displayed as \ldots}\\[5pt]
\indent I am typing a silly paragraph to
see how it will turn out.

And here is the next paragraph.
\end{minipage}

\subsection{Special Characters}
\label{sec:special}
Certain keyboard characters have special significance to \LaTeX, and
you must be aware of how they are used as you're preparing your
document.\index{special characters}  The ten special characters~are:
\begin{verbatim}
# $ % & _ { } ^ ~ \
\end{verbatim}
If you want to use these characters in your text, you cannot enter them in
your manuscript as is: they need extra attention.  The first seven may
be simply prefixed with the backslash character, and the last three
require a small bit of additional syntax.  Here's how you would type
the characters in your document to have them printed:
\begin{verbatim}
\# \$ \% \& \_ \{ \} \^{} \~{} \textbackslash
\end{verbatim}
which, after processing, results in: \# \$ \% \& \_ \{ \} \^{} \~{} \textbackslash

\subsection{Dashes and Quotes}
Since \LaTeX\ is a high-quality typesetting system, we are given
additional choices in preparing our document for output.  For example,
there are four distinct dash-type\index{dash-type characters}
characters available to us, and we
should choose the correct dash for the circumstance:
\begin{center}
\begin{tabular}{c l l}
\textit{Dash Type}&\textit{Typed as \ldots}&\textit{Displayed as \ldots}\\[3pt]
hyphen & \verb|anti-theoretical| & anti-theoretical \\
N-dash & \verb|pages 23--45| & pages 23--45 \\
M-dash & \verb|Wait---I'm thinking| & Wait---I'm thinking \\
minus & \verb|$x-y=z$| & $x-y=z$ \\
\end{tabular}
\end{center}

You should also be aware of single- and double-quote characters: in
well-printed documents, the lefthand-side quotes\index{quote
  characters} are shaped differently than righthand-side quotes.  The
\LaTeX-aware editors that come with the distributions usually handle
this for you automatically when you type the double-quote character
(\texttt{"}) on your keyboard, translating it into the appropriate
pair of characters.

Single quotes don't usually get the same automatic treatment, and you
should type the individual characters yourself.  The left-side
``backward'' single quote character is usually found beneath the tilde
(``squiggle'') on your keyboard.  The right-side single quote
character (apostrophe) is usually found underneath the double-quote
character.  Sometimes it may be necessary to add a small amount of
space between sets of quotes using the \verb|\,| command (as in the
last example below), but this is a more subtle typographic nicety.
\begin{center}
\begin{tabular}{l l}
\textit{Typed as \ldots}&\textit{Displayed as \ldots}\\[3pt]
\verb|This `word' is single-quoted| & This `word' is single-quoted \\
\verb|``These words'' are double-quoted| & ``These words'' are double-quoted \\
\verb|``\,`I' am, I think''| & ``\,`I' am, I think''\\
\end{tabular}
\end{center}

\subsection{Macros, Comments, and Ties}
\label{sec:comments}
\LaTeX{} macros begin with a `\verb+\+' (backslash) special character,
followed by the command name.  Depending on the command,
macros\index{macros}\index{commands|see{macros}} will often require
one or more arguments, and some will accept additional optional
arguments.  The template for such \LaTeX\ commands~is:
\begin{verbatim}
\commandname[option1,option2]{Required Argument}
\end{verbatim}
The optional arguments are included in square brackets (e.g.,
\verb+[+\textit{option}\verb+]+) immediately following the macro
invocation.  Required arguments follow the optional arguments (if any)
between curly braces (e.g., \verb+{+\textit{This is a required
  argument}\verb+}+).  The letter case of the macro is important:
you must spell commands exactly as they are presented, or \LaTeX\ will
complain about unrecognized commands when your document is processed.

The function of the percent sign (\verb|%|) special character is to
introduce a document comment\index{comments}, which runs to the end of
the line of the input file.  Commented text is ignored by \LaTeX{}
entirely, and will not be typeset.  Recall that we may avoid this
special behavior by prefixing a backslash character:
\begin{center}
\begin{tabular}{l l}
\textit{Typed as \ldots}&\textit{Displayed as \ldots}\\[3pt]
\verb|total is 23\% of adjusted gross|&
total is 23\% of adjusted gross\\
\end{tabular}
\end{center}
(Refer to section~\ref{sec:special} for the complete list of special
characters and how to type them.)

\LaTeX\ attempts to fill up each line of a paragraph optimally, and
sometimes we need to provide some advice for where \emph{not} to
create a line break.  The function of the tilde (\verb|~|) special
character is to tie adjoining words together with an ``unbreakable
space''\index{space!unbreakable}\index{unbreakable space} so that they
are not split between lines.  This is often useful to ensure that
honorific titles don't get separated from their name (e.g.,
``\verb|Frau.~Blucher|'', ``\verb|Dr.~Frankenstein|''), or after
certain abbreviations (e.g., ``\verb|a vs.~b|'',
``\verb|cf.~Fig.~5|''), or where numbered entities are referenced,
(e.g., ``\verb|page~23|'').

\subsection{Document Divisions} 
A \LaTeX\ document begins with a division called the
``preamble''\index{preamble}.  In this section, you specify the
document type, set up the document-wide processing settings (like page
margins, or selecting the font size, for example), set the document's
title, author, and other such parameters, and perhaps load additional
packages which provide new features for processing your document.

The rest of the document is called the document ``body''.  It begins
with the \LaTeX\ command
\verb|\begin{document}|\index{environment!document@\verb+document+},
and anything that comes before this command is considered part of the
preamble. Some \LaTeX\ commands are only allowed in the preamble,
while others are allowed only in the document body.

\section{The \fsuth{} Class}
\label{sec:class}

A standard \LaTeX{} installation comes with several pre-defined
document types (called classes), such as \pkg{article},
\pkg{book}, and \pkg{report}.  The \fsuth{} class is an
extension of the \LaTeX{} \pkg{report} class.  In essence, the
\fsuth{} class provides all the features of \pkg{report}, along
with customizations to meet the standards of FSU's
\booktitle{Guidelines \& Requirements for Electronic Theses, Treatises
  and Dissertations}, 2015--2016 edition.  The rest of this
document describes how to use the features of the \fsuth{} class, and
introduces a few other functions provided by \LaTeX.

\subsection{Files in the Package}

Packaged along with this \booktitle{User's Guide} and the \fsuth{}
class file, you will find a folder called
\complit{thesis-template}\index{fsuthesis@\pkg{fsuthesis}!package!contents}.
Within the folder is a small collection of files, a skeleton upon
which you may build your own document.  I suggest that you copy and
rename this folder in a new location, giving your \emph{magnum opus}
its own workspace.

For now, we'll assume that you have renamed the folder
\complit{thesis}.  Inside the folder, you'll find a file called
\complit{mythesis.tex}.  This will be your document's principal file.
We will assume that you will create additional files in this folder to
add to your document, assuming at least one file per chapter.  You are
free to rename any of these files as you like, as long as they end
with the \complit{.tex} extension.

The document skeleton constitutes a complete document as it stands,
and you may run \LaTeX\ on \complit{mythesis.tex} immediately if you
need to test your installation.  (How you invoke \LaTeX\ is
platform-dependent, so you may need to refer to
section~\ref{sec:install} on \textsc{Installation} above for
references specific to your environment.  In the environments provided
by \MiKTeX\ and \MacTeX, processing your document is usually
accomplished by a single button click.)

Besides the \complit{thesis-template} folder, there is a
\complit{sample} folder which contains a more robust sample document,
demonstrating features of \LaTeX\ and the \fsuth\ class, as well as a
few optional features which you may enable if you find them helpful.

\subsection{The Document Preamble}

If you look at the file \complit{mythesis.tex}, you will see that it
consists primarily of \LaTeX{} macros and ``commented out'' lines
containing more \LaTeX{} macros.  (Recall section~\ref{sec:comments}.)
As you add text and flesh out your document, you may ``uncomment''
additional lines in this primary file by removing the leading percent
sign, thereby making the line active.

The document setup may look intimidating at first, but don't let this
deter you.  The template document and the sample document provide you
with some boiler-plate information: you can just fill in the blanks
with your own data to get started quickly.  You don't have to know or
understand all this stuff at first.  I encourage you to look at the
example files and review the output documents to see how they are
correlated.

\subsubsection{Document Options}
\label{sec:docoptions}
The first line of every \LaTeX{} document declares the type of
document to be processed, along with a few processing options.  After
some initial comments, the first line of the document skeleton file
\complit{mythesis.tex} contains the
following:\index{documentclass@\verb=\documentclass=}
\begin{ltxexample}
\documentclass[11pt,expanded]{fsuthesis}
\end{ltxexample}
This line declares the document type to be \fsuth, and that the text
will be set in 11-point type using \verb|expanded| spacing. (Note the
optional arguments supplied in square brackets, and required argument
provided in curly braces.)

Class \fsuth{} is derived from the \pkg{report} class, so all
the standard document options supported by \pkg{report} will be
supported by \fsuth.  (See one of the \LaTeX{} references in
section~\ref{sec:references} for complete lists of document options.)
The \fsuth{} class provides four additional document options\index{documentclass@\verb=\documentclass=!options}:
\complit{hardcopy}, \complit{chapterleaders}, \complit{expanded}, and
\complit{copyright}.

\begin{macrolist}
\item[10pt, 11pt, 12pt] These options select the font point size for
  the document.  Without any\index{font size}\index{point size}
  specifications, \complit{10pt} is the default.  (The standard
  \LaTeX\ \pkg{report} class supplies these options.)

\optionitems{chapterleaders}
This option adds leader dots on chapter headings in the \textsl{Table
  of Contents}.
\index{chapterleaders@\verb+chapterleaders+}%
\index{contents, table of!leaders}%
Normally, chapter headings are displayed in bold type with a page
number and \emph{without} leader dots, while by default, sections and
subsections are displayed with leader dots connecting their page
numbers.  If you write a thesis without sections or subsections, or if
you suppress their display in the \textsl{Table of Contents}, then you
might want to specify the \complit{chapterleaders} option.

\optionitems{copyright}
This option adds a copyright\index{copyright@\verb+copyright+}
statement at the bottom of the title page.  Though your thesis or
dissertation is protected by copyright law already, you may wish to
state the copyright ownership explicitly using this option.

\optionitems{expanded}
This option expands line
spacing\index{expanded@\verb+expanded+}\index{line spacing, expanded} by 50\%.
Some colleges, schools, or departments will prefer expanded spacing to
allow committee members to pencil in comments.  In addition, ETD
requires that the document \emph{not} be single-spaced.

\optionitems{hardcopy}
This option adds extra space along the binding edge of a page.  This
may be useful for printing hard copies\index{hardcopy@\verb+hardcopy+}
for review by your thesis committee, or if you want to have a
professionally bound copy of your thesis or dissertation.  If you also
include the standard \pkg{report} option
\complit{twoside}\index{twoside@\verb+twoside+}, then in addition to
the binding-edge offset, all the chapters of your document will be
forced to start on odd-numbered (right-hand) pages.

\end{macrolist}

\subsubsection{Thesis/Dissertation Description Macros}
The next section in \complit{mythesis.tex} contains several macros
that customize the title page and committee page of your document.  As
a general rule, these macros require text arguments that should be
given in mixed case using title capitalization\index{capitalization}
rules (i.e., each word capitalized, except for articles, prepositions,
and conjunctions; refer to your discipline's preferred style guide if
in doubt).  All proper names should be capitalized normally.  If the
FSU \booktitle{Guidelines} require elements to be displayed
differently (all-caps, for example), the \fsuth{} class will make the
adjustments required for you.  These macros all belong in the
preamble.

\begin{macrolist}
\macroitems{title}
This macro declares the title\index{title@\verb+\title+} of your
thesis or dissertation.  If the title is long, it will be broken over
several lines on the title page.  You can control how the title
is broken into lines\index{title!line breaks} two ways:
using the \LaTeX\ line-separator operator (`\verb+\\+') to force
a line break, or using the word tie (`\verb+~+') between words
to prevent a line break.  (The line-separator command is what the
\LaTeX{} manual calls ``fragile'', and so you must say
`\verb+\protect\\+' when it is used in the argument of the
\verb|\title| command. See the example in the box below.)

\macroitems{author}
This macro stores your name\index{author@\verb+\author+}.  Your name
should be given as specified in the FSU \booktitle{Guidelines}.

\macroitems{college}
This macro should contain the official name of your
school\index{college@\verb+\college+} or college.

\macroitems{department}
If your degree comes from a school or college with separate academic
departments\index{department@\verb+\department+} which issue degrees,
the \verb|\department| macro should declare this name.  Otherwise, you
should comment-out or delete the \verb|\department| line from your
document file.

\macroitems{manuscripttype}
This should be set to one of the following words, as
appropriate:\index{manuscripttype@\verb+\manuscripttype+}
\complit{Thesis}, \complit{Treatise}, or \complit{Dissertation}.

\macroitems{degree}
The title of your degree\index{degree@\verb+\degree+} (e.g., ``Master
of Arts'' or ``Doctor of Philosophy'') is given by the \verb|\degree|
macro.

\macroitems{degreeyear}
The year your degree\index{degreeyear@\verb+\degreeyear+} is awarded
should be set here.  This must be a full 4-digit year.

\macroitems{defensedate}
Use the \verb|\defensedate| macro to specify the
date\index{defensedate@\verb+\defensedate+} of your thesis,
treatise, or dissertation defense.  Refer to the FSU
\booktitle{Guidelines} for the appropriate format.

\end{macrolist}

\noindent 
In your own document up to this point, you could have something like the
following (after the initial \verb|\documentclass| command):
\begin{ltxexample}
\title{This Is My Title:\protect\\ And This Is Its Second Line}
\author{Viktor Spoyles}
\college{College of Arts and Sciences}
\department{Department of Physics}
\manuscripttype{Dissertation}
\degree{Doctor of Philosophy}
\degreeyear{2015}
\defensedate{October 31, 2015}
\end{ltxexample}

If you are generating a \acro{PDF} file, you can add a subject and
search keywords to the document's ``metadata''.\index{metadata} This
information is not printed in your document at all, but it becomes
part of the electronic version of the document's internal
description. The title and author's name will already be included in
the metadata by default.  Since the document metadata are searchable,
adding keywords and a subject may assist people who may be looking
for research like yours.
\begin{macrolist}
\macroitems{subject}
A terse description\index{subject@\verb+\subject+} of the manuscript
topic.

\macroitems{keywords}
A comma- or semi-colon-separated list of germane
keywords\index{keywords@\verb+\keywords+}.

\begin{ltxexample}
\subject{FTL Propulsion Theory}
\keywords{warp drive; wormhole travel; subspace geometry}
\end{ltxexample}

\end{macrolist}

\subsubsection{Committee Macros}

\margcmds{committeeperson}
The \fsuth{} class provides macros for generating your committee
information page.  The
\verb|\committeeperson|\index{committeeperson@\verb+\committeeperson+}
macro takes two arguments.  The first argument is the name of the
committee member, given without titles.  The second argument is the
committee membership status, e.g., ``Professor Directing
Dissertation'' or ``Committee Member''.  (See the FSU
\booktitle{Guidelines} about the appropriate options.)  You should
provide one \verb|\committeeperson| line for each person, in the order
in which they should appear on the committee page.
\begin{ltxexample}
\committeeperson{Arthur Vandelay}{Professor Directing Thesis}
\committeeperson{Whoopsie Daisy}{Committee Member}
\committeeperson{Garnet G. Adirolf}{Committee Member}
\end{ltxexample}

\subsection{The Document Body}

\margenvs{document}
With the document setup complete, you start the document body with the
\LaTeX{} command
\verb|\begin{document}|.\index{environment!document@\verb+document+}
You will notice that whenever you
\verb|\begin{|\textit{something}\verb|}|, you should always supply a
corresponding \verb|\end{|\textit{something}\verb|}| or \LaTeX{} will
complain.  So at the end of \complit{mythesis.tex}, you will find the
\verb|\end{document}| command.  Anything beyond this point in the file
is ignored by \LaTeX{}.  In \LaTeX\ parlance, anything enclosed within
a
\verb|\begin{|\textit{something}\verb|}|\,\ldots\verb|\end{|\textit{something}\verb|}|
pair is called an ``environment''.\index{environment} We'll encounter
several environments along the way.

\subsubsection{Front Matter}

\begin{macrolist}
\macroitems{frontmatter}
The first element after \verb|\begin{document}| should be the macro
command \verb|\frontmatter|,\index{frontmatter@\verb+\frontmatter+}
which sets up roman numeral\index{roman numerals} page numbering for
the document elements that precede the first chapter of your thesis or
dissertation.  The document skeleton in \complit{mythesis.tex}
contains place-holders in the proper order for all the optional
elements of the front matter.  Uncomment those elements that you will
use, or you may leave commented or delete those elements that you
don't use.

\macroitems{maketitle,makecommitteepage}
Once \verb|\frontmatter| has set the stage, the macro command
\verb|\maketitle|\index{maketitle@\verb+\maketitle+} will generate the
document title page.  Likewise, the
\verb|\makecommitteepage|\index{makecommitteepage@\verb+\makecommitteepage+}
macro will create the committee page.  Information for these pages is
gathered from the data you have already set in macro calls in the
preamble.
\begin{ltxexample}
\begin{document}
\frontmatter
\maketitle
\makecommitteepage
\end{ltxexample}

\envitems{dedication}
If you wish to include an optional dedication in your thesis or
dissertation, uncomment the
\verb|\begin{dedication}|
\index{environment!dedication@\verb+dedication+}%
\index{dedication@\verb+dedication+}%
and \verb|\end{dedication}|
lines, and type your dedication between them.  The text that you
insert will appear about 1/3rd of the distance from the top of the
page.  The rest of the formatting is up to you.
\begin{ltxexample}
\begin{dedication}
  \begin{center}
    To my parents
  \end{center}
\end{dedication}
\end{ltxexample}

\envitems{acknowledgments}
Likewise, if you wish to include acknowledgments in your document,
uncomment the
\verb|\begin{acknowledgments}|
\index{acknowledgments@\verb+acknowledgments+}%
\index{environment!acknowledgments@\verb+acknowledgments+}%
and \verb|\end{acknowledgments}| lines, and insert the acknowledgment text
between these lines.  The resulting page will have the centered
heading \textbf{ACKNOWLEDGMENTS}, followed by your text.
\begin{ltxexample}
\begin{acknowledgments}
Thanks to my committee, especially my major professor.
\end{acknowledgments}
\end{ltxexample}

\macroitems{tableofcontents}
The next item in the front matter is the \textsl{Table of Contents},
which is generated for you automatically by the macro
\verb|\tableofcontents|.
\index{tableofcontents@\verb+\tableofcontents+}%
\index{contents, table of}%
By default, the \textsl{Contents} page(s)
will contain entries for the remaining front matter material, and
entries for chapter headings, section headings, and subsection
headings.  If you find your \textsl{Table of Contents} has too much
detail, you may adjust the level of headings included.  (See
section~\ref{sec:tocdetail}.) 

\macroitems{listoftables,listoffigures,listofmusex}
The FSU \booktitle{Guidelines} state that if you have more than one
figure
\index{listoftables@\verb+\listoftables+}%
\index{listoffigures@\verb+\listoffigures+}%
\index{listofmusex@\verb+\listofmusex+}%
or table in your document, the figures and/or tables should be
contained in their own lists.  Turn each of these options on by
uncommenting the \verb|\listoftables| and/or \verb|\listoffigures|
lines in \complit{mythesis.tex}.  These tables will be generated for
you automatically when your document is processed.  For those
documents which contain multiple musical examples, a list of these may
also be generated by uncommenting \verb|\listofmusex|.
\begin{ltxexample}
\tableofcontents
\listoftables
\listoffigures
\listofmusex
\end{ltxexample}

\envitems{listofsymbols,listofabbrevs} If a \textsl{List of Symbols}
or \textsl{List of Abbreviations} might be helpful to your readers,
\index{environment!listofsymbols@\verb+listofsymbols+}%
\index{environment!listofabbrevs@\verb+listofabbrevs+}%
\index{listofabbrevs@\verb+listofabbrevs+}%
\index{listofsymbols@\verb+listofsymbols+}%
\fsuth{} provides these environments.  If you wish to include such
document elements, uncomment the appropriate
\verb|\begin|~\ldots\verb|\end| pair, and add any text you may
require.

These entities would likely consist of tabular material, so you'll
want to dig into \LaTeX{} table-making using any of the basic
references mentioned in section~\ref{sec:references}.  Below is a
simple example of how you might use these environments to create
tables with some useful information if needed.
\begin{ltxexample}
\begin{listofsymbols}
  \begin{center}\begin{tabular}{r l}
    $E$   & Energy--Mass equivalence: $mc^2$ \\
    $R_e$ & Mean Radius of the Earth, ${}\approx 6367.65\,\textup{km}$\\
    $\pi$ & $3.1415926\ldots$\\
  \end{tabular}\end{center}
\end{listofsymbols}
\end{ltxexample}
\begin{ltxexample}
\begin{listofabbrevs}
  \begin{center}\begin{tabular}{l l}
    i.e. & \textit{id est}, ``that is''\\
    e.g. & \textit{exempli gratia}, ``for example''\\
  \end{tabular}\end{center}
\end{listofabbrevs}
\end{ltxexample}

\envitems{abstract}
The last element of the front matter is a document abstract.
\index{environment!abstract@\verb+abstract+}%
\index{abstract@\verb+abstract+}%
Insert your text between the abstract \verb|\begin|~\ldots\verb|\end| pair.
The proper heading is included for you automatically.
\begin{ltxexample}
\begin{abstract}
This abstract is a concrete example.
\end{abstract}
\end{ltxexample}

\end{macrolist}

\subsection{The Main Text}

\margcmds{mainmatter}
At last, with the preliminaries out of the way, you may now get to the
meat of your document.  Following the abstract, the command
\verb|\mainmatter|\index{mainmatter@\verb+\mainmatter+} restarts page
numbering at ``1'' in arabic numerals\index{arabic numerals}, ready
for your first chapter.

The skeleton file \complit{mythesis.tex} has been set up to include
the first chapter from an external file.  Note the
command \verb+\input+:\index{input@\verb+\input+}
\begin{verbatim}
\input chapter1
\end{verbatim}
This tells \LaTeX{} to insert the text of the file
\complit{chapter1.tex} at this position and continue processing. (The
file extension is appended automatically.)  There is nothing special
about the file names\index{file names}, except that they should end
with the extension \complit{.tex}.  Otherwise, you may call the
external files whatever you like.  (However, avoid using filenames
with spaces or special symbols, as these may be difficult for either
\LaTeX{} or your operating system to handle properly.)  You can break
up large chapters into even smaller pieces if you like, and then
change \complit{mythesis.tex} accordingly, e.g.,
\begin{verbatim}
\input chapter1a
\input chapter1b
\end{verbatim}
Or you could just continue adding text to \complit{mythesis.tex}
directly, avoiding having to deal with any other external files
entirely.  This is all up to you.

\subsubsection{Chapter and Section Headings}

\margcmds{chapter,section,subsection,subsubsection,paragraph,subparagraph}
Several levels of headings
are provided by the \fsuth{} class in the
heading styles defined by FSU's \booktitle{Guidelines}.
\index{chapter@\verb+\chapter+}%
\index{section@\verb+\section+}%
\index{subsection@\verb+\subsection+}%
\index{subsubsection@\verb+\subsubsection+}%
\index{paragraph@\verb+\paragraph+}%
\index{subparagraph@\verb+\subparagraph+}%
\index{headings}%
By default,
entries down to the subsection level are listed in the \textsl{Table
  of Contents}.  (See section~\ref{sec:tocdetail} for information on
changing this default.) Listed from the highest level downward, these
sectioning commands are:
\begin{itemize}\addtolength{\parskip}{-7pt}
\item \verb|\chapter{My Chapter Title}|
\item \verb|\section{A Main Section Heading}|
\item \verb|\subsection{The Subsection Heading}|
\item \verb|\subsubsection{A Subsubsection Title}|
\item \verb|\paragraph{Do You Need This Many Levels?}|
\item \verb|\subparagraph{Come On, You're Kidding!}|
\end{itemize}
Each of these macros takes a single argument, the text of the heading.
All headings should be capitalized\index{capitalization} as titles,
i.e., mixed case text, each word capitalized except articles,
prepositions, and conjunctions.  Chapter headings will force the start
of a new page.  The file \complit{chapter1.tex} in the
\complit{thesis} folder has some example text to get you started.  (If
any of your headings include math symbols, you may want to activate
the \pkg{textcase} package. See section~\ref{sec:packages} for more
information.)

By default, section and subsection headings are prefixed by section
and subsection numbers. Sub-subsections produce an unnumbered in-line
heading in bold-face type as the opening of a paragraph.  Paragraph and
sub-paragraph headings also produce in-line headings and start new
paragraphs, but with progressively subtler font selections.
If you like, you may change the level at which headings are numbered.
See section~\ref{sec:headnum} for more information.

\subsection{Back Matter}

\margcmds{appendix}
Following the major chapters of your manuscript, you may have
additional material for one appendix\index{appendix@\verb+\appendix+}
or more.  To shift from chapter headings to appendix headings, insert
the macro \verb+\appendix+ at the end of your last chapter, before the
first appendix.  Then use the \verb+\chapter+ macro just as you have
for each of your chapters.  (Appendices will be lettered rather than
numbered.)
\begin{ltxexample}
\appendix
\input appendix1
\input appendix2
\end{ltxexample}

\subsubsection{References/Bibliography}

\margenvs{references}
The \fsuth{} class provides two options to produce a bibliography or
references section.  The first (and simplest) option is to use the
\complit{references}
environment.
\index{environment!references@\verb+references+}%
\index{references@\verb+references+}%
Begin this section with
\verb|\begin{references}|.  Then add each bibliographic entry with a
blank line between each reference.  Follow the last entry with
\verb+\end{references}+.  With this option, you will have to format
each entry according to the style guide you have chosen to follow.
\begin{ltxexample}
\begin{references}
Picaut, J., F. Masia, and Y. du Penhoat, 1997: An advective--reflective
conceptual model for the oscillatory nature of the ENSO.
\textit{Science}, \textbf{277}, 663--666.

Yasunari, T., 1990: Impact of Indian monsoon on the coupled
atmosphere/ocean system in the tropical Pacific.
\textit{Meteor. Atmos. Phys.}, \textbf{44}, 19--41.
\end{references}
\end{ltxexample}
This example provides two bibliography entries.  I had to specify the
style-guide preferences myself (such as presenting journal names in italic
type, volume numbers in bold-face, etc.).  Once these are processed,
they'll look like the following:
\begin{narrowref}
Picaut, J., F. Masia, and Y. du Penhoat, 1997: An advective--reflective
conceptual model for the oscillatory nature of the ENSO.
\textit{Science}, \textbf{277}, 663--666.

Yasunari, T., 1990: Impact of Indian monsoon on the coupled
atmosphere/ocean system in the tropical Pacific.
\textit{Meteor. Atmos. Phys.}, \textbf{44}, 19--41.
\end{narrowref}

\margcmds{cite,bibliographystyle,bibliography}
The second option is to set up a \BibTeX{}\index{bibtex@\BibTeX}
database.
\index{cite@\verb+\cite+}%
\index{bibliographystyle@\verb+\bibliographystyle+}%
\index{bibliography@\verb+\bibliography+}%
To use \BibTeX, you create
or download a separate file of reference materials in a particular
format.  Then you may cite any of these references within your
manuscript using the \verb|\cite| macro.  By running \LaTeX{} in
combination with \BibTeX, citations are resolved, and a list of the
cited references are pulled into your document automatically.  To use
this feature, you first select the bibliographic style, and then
specify the \BibTeX{} database file:
\begin{ltxexample}
\bibliographystyle{plain}
\bibliography{myrefs}
\end{ltxexample}
This selects the \complit{plain} bibliography style, and the
\BibTeX{} database is said to reside in \complit{myrefs.bib}.
Processing your document now requires a few extra steps as well:
\begin{itemize}\addtolength{\parskip}{-7pt}
\item Run \LaTeX{}\par\nobreak
\item Run \BibTeX{}\par\nobreak
\item Run \LaTeX{} \emph{twice more}
\end{itemize}

If you have a relatively small number of bibliographic entries or
citations, then choosing the \complit{references} environment is
probably the easiest solution.  However, if you are trying to manage a
large number of citations or work in a discipline that has already
established a large \BibTeX{} database, then it may save you
considerable effort to learn how to use \BibTeX, in which case you
will certainly need to use one of the \LaTeX{} references mentioned
earlier.

Depending on your discipline, you may need more substantial citation
and bibliography-formatting capabilities.  Many disciplines and
journals have created their own \BibTeX\ styles, and you may want to
take advantage of these efforts.  First, you'll need to install the
correct option package (see section~\ref{sec:packages} to get
started).  You should also check out the document in the
\complit{sample} directory for some more concrete examples.

\subsubsection{Biographical Sketch}

\margenvs{biosketch}
At last, you've reached the final page of your \booktitle{magnum
  opus}.  It will contain your biographical sketch, starting with
\verb+\begin{biosketch}+,
\index{environment!biosketch@\verb+biosketch+}%
\index{biosketch@\verb+\biosketch+}%
and ending with \verb+\end{biosketch}+ as usual.  Insert what
biographical material you wish to include here, but remember not to
include any personal contact information.

Once the biographical sketch is done, we wrap things up with the
\verb|\end{document}| command (matching the \verb|\begin{document}|
earlier in the file).  Anything that follows \verb|\end{document}|
will be ignored by \LaTeX.
\begin{ltxexample}
\begin{biosketch}
The author was born, educated, and re-educated
until this document was finally completed.
\end{biosketch}

\end{document}
\end{ltxexample}

\section{More \LaTeX\ Features}

\subsection{List Environments}

\margenvs{itemize,enumerate,description}
Standard \LaTeX\ provides a means of generating lists of things.
\index{lists}%
\index{environment!itemize@\verb+itemize+}%
\index{environment!enumerate@\verb+enumerate+}%
\index{environment!description@\verb+description+}%
\index{itemize@\verb+itemize+}%
\index{enumerate@\verb+enumerate+}%
\index{description@\verb+description+}%
A simple bulleted list can be created this way:

\medskip
\noindent\hskip2em
\index{lists!bulleted}\begin{minipage}[t]{2.1in}
\textit{Typed as \ldots}
\end{minipage}
\hskip3em
\begin{minipage}[t]{2.5in}
\textit{Displayed as \ldots}
\end{minipage}
\par\nobreak\vskip3pt\nobreak
\noindent\hskip2em
\begin{minipage}[t]{2.1in}%
\small\begin{verbatim}
\begin{itemize}
\item Granny Smith Apples
\item Hominy Grits
\item Gruy\`ere Cheese
\end{itemize}
\end{verbatim}
\end{minipage}
\hskip3em
\begin{minipage}[t]{2.5in}\small\sloppy
\begin{itemize}
\item Granny Smith Apples
\item Hominy Grits
\item Gruy\`ere Cheese
\end{itemize}
\end{minipage}
\bigskip

Similarly, an enumerated list is constructed this way:

\medskip
\noindent\hskip2em
\index{lists!enumerated}\begin{minipage}[t]{2.1in}
\textit{Typed as \ldots}
\end{minipage}
\hskip3em
\begin{minipage}[t]{2.5in}
\textit{Displayed as \ldots}
\end{minipage}
\par\nobreak\vskip3pt\nobreak
\noindent\hskip2em
\begin{minipage}[t]{2.1in}%
\small\begin{verbatim}
\begin{enumerate}
\item Beat eggs until frothy.
\item Stir in milk.
\item Whisk in flour.
\end{enumerate}
\end{verbatim}
\end{minipage}
\hskip3em
\begin{minipage}[t]{2.5in}\small\vskip-5pt
\begin{enumerate}
\item Beat eggs until frothy.
\item Stir in milk.
\item Whisk in flour.
\end{enumerate}
\end{minipage}
\bigskip

Yet another list style is provided by the \complit{description}
environment.  This looks like the following:

\medskip
\noindent\hskip2em
\index{lists!description}\begin{minipage}[t]{2.1in}
\textit{Typed as \ldots}
\end{minipage}
\hskip3em
\begin{minipage}[t]{2.5in}
\textit{Displayed as \ldots}
\end{minipage}
\par\nobreak\vskip3pt\nobreak
\noindent\hskip2em
\begin{minipage}[t]{2.1in}%
\small\begin{verbatim}
\begin{description}
\item[Green eggs] A delicacy.
\item[Ham] Cured pork.
\item[Sam I Am] Hawker of food.
\end{description}
\end{verbatim}
\end{minipage}
\hskip3em
\begin{minipage}[t]{2.5in}\small
\begin{description}
\item[\defaultcolor Green eggs] A delicacy.
\item[\defaultcolor Ham] Cured pork.
\item[\defaultcolor Sam I Am] Hawker of food.
\end{description}
\end{minipage}
\bigskip

You may also create lists inside of lists, mixing and matching
\complit{itemize}, \complit{enumerate}, and \complit{description}
styles as necessary.  Lists may also be customized in many ways, so if
you need something fancier, you should read through one of the references.

\subsection{Quotation Environments}

\margenvs{quote,quotation}
\LaTeX\ provides for another common construction in documents:
quotations.\index{quotations} Quotations are set off from the rest of
the text with narrower margins (and smaller line-spacing if the
\complit{expanded} option has been turned on).  There are two
varieties of quotation environments.  For shorter quotations, use the
\complit{quote} environment.  Each paragraph in the \complit{quote}
\index{environment!quote@\verb+quote+}%
\index{quote@\verb+quote+}%
environment is not indented (though the
margins are narrower), and there's just a small space added between
them.  This is appropriate for one or two-line quotes, or for a series
of short quotations.

\medskip
\noindent\hskip2em
\begin{minipage}[t]{2.1in}
\textit{Typed as \ldots}
\end{minipage}
\hskip3em
\begin{minipage}[t]{2.5in}
\textit{Displayed as \ldots}
\end{minipage}
\par\nobreak\vskip3pt\nobreak
\noindent\hskip2em
\begin{minipage}[t]{2.1in}%
\small\begin{verbatim}
\begin{quote}
Ars Longa, Vita Brevis.

Wow, that's a short quotation.
\end{quote}
\end{verbatim}
\end{minipage}
\hskip3em
\begin{minipage}[t]{2.5in}\small
\begin{quote}
Ars Longa, Vita Brevis.

Wow, that's a short quotation.
\end{quote}
\end{minipage}
\bigskip

For longer quoted passages (one or more paragraphs), use the
\complit{quotation}
\index{environment!quotation@\verb+quotation+}%
\index{quotation@\verb+quotation+}%
environment.  Each paragraph in the \complit{quotation} environment is
indented.

\medskip
\noindent\hskip2em
\begin{minipage}[t]{2.1in}
\textit{Typed as \ldots}
\end{minipage}
\hskip3em
\begin{minipage}[t]{2.5in}
\textit{Displayed as \ldots}
\end{minipage}
\par\nobreak\vskip3pt\nobreak
\noindent\hskip2em
\begin{minipage}[t]{2.1in}%
\small\begin{verbatim}
\begin{quotation}
This quotation is a bit
longer, just as a test.

And here's a second
paragraph now.
\end{quotation}
\end{verbatim}
\end{minipage}
\hskip3em
\begin{minipage}[t]{2.5in}\small
\begin{quotation}
This quotation is a bit
longer, just as a test.

And here's a second
paragraph now.
\end{quotation}
\end{minipage}
\bigskip

\subsection{Single-Spacing Environment}

\margenvs{singlespaced}
If our document is being typeset in \complit{expanded} spacing mode,
\index{singlespaced@\verb+singlespaced+}%
\index{environment!singlespaced@\verb+singlespaced+}%
\index{line spacing, single}%
there may be times when we prefer to typeset something in
single-spaced mode.  For this purpose, the \pkg{fsuthesis} class
provides the \complit{singlespaced} environment.
\begin{ltxexample}
\begin{singlespaced}
This paragraph will be typeset
in single-space mode, even when I've
asked for expanded spacing.
\end{singlespaced}
\end{ltxexample}
You should not need this environment too often:  lists and quotations
already reduce their spacing for you.  Also, if you are typing
algorithms or computer code in \LaTeX's \complit{verbatim} environment, 
spacing has already been reset to single-space mode.  If you have not
set the \complit{expanded} spacing document option, then this
environment formats its contents as a regular paragraph.

\subsection{Insertions: Figures, Tables, Musical Examples}

\margenvs{figure,table,musex}
Standard \LaTeX\ provides environments for \complit{figure}s and
\complit{table}s.
\index{insertions}%
\index{environment!figure@\verb+figure+}%
\index{environment!table@\verb+table+}%
\index{environment!musex@\verb+musex+}%
\index{figure@\verb+figure+}%
\index{table@\verb+table+}%
\index{musex@\verb+musex+}%
The \pkg{fsuthesis} class provides an additional
environment called \complit{musex} for those authors who need to
provide musical examples.  The \complit{musex} environment behaves
similarly to the \complit{figure} environment, except that captions
include the heading ``Example'' instead of ``Figure'', and all the
musical examples can be listed in the front matter in the \textsl{List
  of Musical Examples}.

By setting material off in a \complit{figure}, \complit{table}, or
\complit{musex} environment, the material will be allowed to drift
from its position in the text to the closest available location as
follows: if there is space for the material at the bottom of the
current page, it will be placed there; otherwise, it will be placed at
the top of the next page, or perhaps on a page by itself.  (You have
some control over the placement of floating elements.  For more
detail, you'll need to consult one of the \LaTeX{} references in
section~\ref{sec:references}.)

\margcmds{caption}
Each \complit{figure}, \complit{table}, or \complit{musex} should
contain a \LaTeX{} \verb+\caption+
command\index{caption@\verb+\caption+} whose single argument contains
the text of the caption.  For figures and musical examples, the
caption should be placed below the figure or musical example.  For
tables, the caption should be located above the tabular material.
Examples of the use of each of these environments can be found in the
in the \complit{sample} directory.

\LaTeX{} keeps track of the number of tables, figures, and musical
examples, and your caption will be labeled and numbered automatically.
The caption text will also be inserted into the appropriate
\mbox{\textsl{List of \ldots}} if you requested the list in the
front matter of your document.

\LaTeX{} has many features to assist you in producing tabular material
of arbitrary complexity.  Also, simple diagrams may be created using
the \LaTeX{} \complit{picture} environment.
\index{picture@\verb+picture+}\index{environment!picture@\verb+picture+}%
If you want to include graphics generated by external
software, then you'll need to learn to use the features of the
\pkg{graphicx} package\index{graphicx@\verb+graphicx+}, and you should
add the appropriate \verb+\usepackage+ command in
\complit{mythesis.tex} preamble (see section~\ref{sec:packages}).  You
are strongly advised to refer to the \LaTeX{} references to learn more
about figures and tables if you intend to use them in your manuscript.

\subsection{Cross References}
\label{sec:labels}
One of the advantages of working with \LaTeX{} is the ability to
cross-reference\index{cross references} equations, figures, tables,
musical examples, and section headings.  In writing and revising your
manuscript, it is likely that references to elements may shift as text
is added or moved around.  \LaTeX{} addresses this problem by allowing
you to assign a \textit{label key} to each element.  Then you make a
reference to an element's label key in your text to retrieve its
number or page location.  When your document is processed, \LaTeX{}
replaces all the label key references with their numerical values.

As an example, let's take a look at how this might work if we wish to
refer to an equation in our text.  The left column is the source text
we've typed (unimportant text omitted for brevity), and the result is
in the right column.

\medskip
\noindent\hskip2em
\index{label@\verb+\label+}\begin{minipage}[t]{2.0in}
\textit{Typed as \ldots}
\end{minipage}
\hskip3em
\begin{minipage}[t]{2.5in}
\textit{Displayed as \ldots}
\end{minipage}
\par\nobreak\vskip3pt\nobreak
\noindent\hskip2em
\begin{minipage}[t]{2.0in}%
\small\begin{verbatim}
Leonhard Euler was a prolific
...
Equation~\ref{eq:euler-id} on
page~\pageref{eq:euler-id} is
...
\begin{equation}
e^{i\pi} + 1 = 0
\label{eq:euler-id}
\end{equation}
\end{verbatim}
\end{minipage}
\hskip3em
\begin{minipage}[t]{2.5in}%
  \small\sloppy
  Leonhard Euler was a prolific mathematician whose
  pioneering work in power series helped to develop the field
  of mathematical analysis.  Equation~\ref{eq:euler-id} on
  page~\pageref{eq:euler-id} is known as
  \textit{Euler's Identity}, what physicist Richard Feynman called
  ``the most remarkable formula in mathematics''.
  \begin{equation}
    e^{i\pi} + 1 = 0
    \label{eq:euler-id}
  \end{equation}
\end{minipage}
\bigskip

\margcmds{label,ref,pageref}
The \complit{equation} environment
\index{label@\verb+\label+}%
\index{ref@\verb+\ref+}%
\index{pageref@\verb+\pageref+}%
\index{equation@\verb+equation+}%
\index{environment!equation@\verb+equation+}%
automatically numbers equations for us.  The command
\verb|\label{eq:euler-id}| creates the key
``\complit{eq:euler-id}'', tying it to the equation
number.  To access the equation number, we use the
\verb|\ref{eq:euler-id}| macro, while the macro
\verb|\pageref{eq:euler-id}| retrieves the page number.  For figures,
tables, or musical examples, the \verb|\label| command should
immediately follow the \verb|\caption| macro.  In the example, you
might also note the use of the tie special character between the
\verb|\ref| and \verb|\pageref| commands and the text that precedes
them (see section~\ref{sec:comments}).

Since your text may \verb|\ref| label keys before the corresponding
\verb|\label| has been encountered, you will need to run your document
through the \LaTeX{} processor \emph{at least twice}.  The first pass
will write all the label keys and page numbers out to an auxiliary
file, and the second pass will then be able to resolve all the
references properly.  (\LaTeX{} will complain about unresolved or
changed references, reminding you to run the processor a second time.)

As you're writing your document, you might want to keep a list of the
label keys you've created so that you don't have to surf through other
files to recall what a particular label key was.  Keep in mind that
figures, tables, musical examples, and equations all use the same
label system, and all label keys must be unique.  You may develop your
own label key standards (like using \complit{eq:} when referencing an
equation, \complit{fig:} to label a figure, etc., to avoid label
``collision'').  If you expect to have lots of figures, tables, etc.,
you may find it helpful to use descriptive label keys rather than
generic ones, as they may be easier to remember.  E.g.,
\complit{fig:map-Europe-pre1914} is probably more mnemonic than
\complit{fig:MapOne}.

\subsection{A Note on Mathematics}
\LaTeX{} excels at mathematical
\index{mathematics}
typesetting, but to get the best
results, practice and experience go a long way.  If you have a lot
of math in your document, you owe it to yourself and to your
audience to read through some of the \LaTeX{} references on the
subject (see section~\ref{sec:references}).

Since I've seen this mistake in many documents, I will highlight
one point here: treat equations as part of a paragraph.
\index{mathematics!spacing}
In terms of typing your document, this means that there should be
no blank lines between text and displayed equations. The example
which follows is a common idiom.

\medskip
\noindent\hskip1.75em
\begin{minipage}[t]{2.4in}
\textit{Typed as \ldots}
\end{minipage}
\hskip1.5em
\begin{minipage}[t]{2.3in}
\textit{Displayed as \ldots}
\end{minipage}
\par\nobreak\vskip3pt\nobreak
\noindent\hskip1.75em
\begin{minipage}[t]{2.4in}%
\small\begin{verbatim}
This equation calculates the
relationship index $r$:
\begin{equation}
r = a(N-1)
    - \left(\sum_{j=1}^a j\right)
    + b - 1,
\end{equation}
where $0 \le a < b < N$, $N$ is the
count of entities, and $a$ and $b$
are entity indices.
\end{verbatim}
\end{minipage}
\hskip1.5em
\begin{minipage}[t]{2.3in}\small
This equation calculates the
relationship index~$r$:
\begin{equation}
r = a(N-1) - \left(\sum_{j=1}^a j\right) + b - 1,
\end{equation}
where $0 \le a < b < N$, $N$ is the
count of entities, and $a$ and $b$ are
entity indices.
\end{minipage}
\bigskip

Blank lines surrounding the equation in the document source
tell \LaTeX{} to leave too much vertical space, disconnecting
it from the paragraph. Worse, the text following the equation
would be indented as if it started a new paragraph, which is
\emph{not} the intention in this case.  Of course, if an
equation actually does begin or end a paragraph, then leave
a blank line before or after, as appropriate.

\section{Changing Default Settings}

You may change document default settings in the document preamble.
For example, should you want to change the width of the text column or
the page margins, the document preamble is where you would do it.
(Note that you must still adhere to FSU's \booktitle{Guidelines}, so
be sure you know what you're doing.)

\subsection{Changing \textsl{Table of Contents} Detail}
\label{sec:tocdetail}
If listing section or subsection headings provides too
much detail
\index{contents, table of!detail}%
\index{tocdepth@\verb+tocdepth+}%
for your taste, you may remove these entries by resetting
the \LaTeX{} counter \complit{tocdepth}.  \LaTeX{} considers chapter
headings to be Level~0, section headings to be Level~1, and so on.
The default setting of \complit{tocdepth} is~2 (so subsection
headings are included).  To include only chapter and section headings
in the \textsl{Contents}, for example, you could reset
\complit{tocdepth} in the document preamble with the following line:
\begin{ltxexample}
\setcounter{tocdepth}{1}
\end{ltxexample}
If you set \complit{tocdepth} to zero, then only chapter titles will
be listed in the table of contents.  In this case, you may want to
enable the \complit{chapterleaders} document option. (See
section~\ref{sec:docoptions}).

\subsection{Changing the Level of Heading Numbering}
\label{sec:headnum}
You may change the level at which the heading macros produce numbered
entries by setting \complit{secnumdepth}.
\index{headings!numbered}%
\index{secnumdepth@\verb+secnumdepth+}%
The default setting is
level~two, which means that headings up to subsections will be
numbered automatically.  To stop numbering at the \complit{section}
level (for example), reduce the value of \complit{secnumdepth} to one
by issuing the following command in the document preamble:
\begin{ltxexample}
\setcounter{secnumdepth}{1}
\end{ltxexample}
By setting \complit{secnumdepth} to zero, you may disable all heading
numbering except at the chapter level.  Or you may increase the value
up to five to generate heading numbers all the way down to the
sub-paragraph heading level.

\subsection{Avoiding Club and Widow Lines}
\LaTeX\ makes some typographic decisions (for example, where to place
a page break) based on a system of penalties that are calculated while
the document is processed.
\index{club lines}%
\index{widow lines}%
\index{widowpenalty@\verb+\widowpenalty+}%
\index{clubpenalty@\verb+\clubpenalty+}%
The \complit{sample} document sets
the penalties for ``widow'' and ``club'' lines\footnote{The last line
  of a paragraph which occurs at the top of a page is called a
  ``widow''; the first line of a paragraph which occurs at the bottom
  of a page is called a club.  These are relatively ugly,
  typographically speaking.} as follows:
\begin{ltxexample}
\widowpenalty=9999
\clubpenalty=9999
\end{ltxexample}
To \LaTeX, the value 10,000 is the maximum penalty.  By setting these
values to almost the maximum, \LaTeX\ will avoid generating widow or
club lines in all but the most extreme cases.

\subsection{Adjustments to Titles}
Any of the headings in your document may be re-labeled, should you
need to do so.\index{headings!default}
The following is a table of heading variables and
the default setting in the \pkg{fsuthesis} class.
\begin{center}\small
\begin{tabular}{l l l l}
\textit{Variable Name}&\textit{Default Text}&
  \textit{Variable Name}&\textit{Default Text}\\[3pt]
\verb|\abstractname| & Abstract\index{abstractname@\verb+\abstractname+} &
   \verb|\listabbrevname| & List of Abbreviations\index{listabbrevname@\verb+\listabbrevname+} \\
\verb|\acknowledgename| & Acknowledgments\index{acknowledgename@\verb+\acknowledgename+} &
  \verb|\listfigurename| & List of Figures\index{listfigurename@\verb+\listfigurename+} \\
\verb|\appendixtocname| & Appendix\index{appendixtocname@\verb+\appendixtocname+} &
  \verb|\listtablename| & List of Tables\index{listtablename@\verb+\listtablename+} \\
\verb|\bibname| & Bibliography\index{bibname@\verb+\bibname+}  &
  \verb|\listmusexname| & List of Musical Examples\index{listmusexname@\verb+\listmusexname+} \\
\verb|\biosketchname| & Biographical Sketch\index{biosketchname@\verb+\biosketchname+} &
  \verb|\listsymname| & List of Symbols\index{listsymname@\verb+\listsymname+} \\
\verb|\contentsname| & Table of Contents\index{contentsname@\verb+\contentsname+} &
  \verb|\musexname| & Example\index{musexname@\verb+\musexname+} \\
\end{tabular}
\end{center}

Should you wish to change any of these defaults~--- for example, you need
to change ``Bibliography'' to ``References''~--- you would add the
following to the document preamble:
\begin{ltxexample}
\renewcommand*{\bibname}{References}
\end{ltxexample}
Similarly, if you have more than one appendix, you may want to change
\verb|\appendixtocname| so that the entry in the table of contents is
correct. 
\begin{ltxexample}
\renewcommand*{\appendixtocname}{Appendices}
\end{ltxexample}

There's another way that this mechanism could be useful.  Let's say
that your document has a lot of maps associated with it, and you'd
like to keep the lists of maps and figures separate.  Further, you'd
like to have a ``List of Maps'' included in the front matter of your
document.  You can do this by renaming some other environment that
you're not using, say the List of Musical Examples.
\begin{ltxexample}
\renewcommand*{\musexname}{Map}
\renewcommand*{\listmusexname}{List of Maps}
\end{ltxexample}
Having done this, whenever you invoke the \complit{musex} environment,
the caption will be labeled with ``Map'' rather than
``Example''.  If you include the \verb|\listofmusex| in the front
matter of your document, then ``List of Maps'' will be the heading of
that page instead of ``List of Musical Examples''.

\subsection{Extra Packages}
\label{sec:packages}
\LaTeX{} has many document feature add-ons.
\index{packages}%
If you wish to load additional packages, these options should follow
the document class selection.  Be warned that some packages may not be
compatible with the \fsuth{} class.  Many optional packages may
already come installed with your \TeX/\LaTeX{} distribution, or you
can download and install them from the \acro{CTAN} website
(\url{http://www.ctan.org}).  The skeleton document does not enable
any additional packages.  However, the \complit{sample} folder
provides several examples of external packages you may wish to use.

You enable features by adding a \verb+\usepackage+
command\index{usepackage@\verb+\usepackage+} to the document preamble.
Some packages provide sets of features or configuration options which
can be enabled via the command line.  Here are some~examples:
\begin{ltxexample}
\usepackage{amsmath}
\usepackage[round]{natbib}
\end{ltxexample}
This loads the \pkg{amsmath} package with its default settings and
options, and it loads the \pkg{natbib} package with the
\complit{round} option (which sets citations off in parentheses rather
than square brackets).  Read the package documentation! Packages may
rely on other packages and options, and some combinations of packages
may be incompatible.

The following list is by no means exhaustive.  There are hundreds of
packages available, but a few may be generally useful (or harmful) to
a wide number of users, and so I highlight them here.
\begin{description}
\item[\pkg{geometry}, \pkg{setspace}] \emph{Do not use these packages
  with \fsuth!} (unless you really know what you're doing).  These
  packages will change page layouts and spacing, undoing a lot of
  the hard work that \fsuth{} has done for you.  In many cases,
  interactions between settings in \pkg{geometry} or \pkg{setspace}
  and \fsuth{} will cause ugly or non-compliant spacing. You have
  been warned!

\item[\pkg{amsmath}] The American Mathematical Society has
  developed a large set of symbols and macros to assist you in
  producing high-quality mathematics.\index{mathematics!package}

\item[\pkg{textcase}] If you have title, chapter, or section
  headings which include mathematic symbols, you may want to
  install the \pkg{textcase} package, as this will prevent the
  titling macros from upper-casing the math symbols inappropriately.

\item[\pkg{graphicx}] If you will be inserting figures into
  your document electronically, you should investigate the
  \pkg{graphicx} package.  You can find some simple examples of
  figure inclusion in the \complit{sample} directory, but for the
  highest quality output, you owe it to yourself to learn more about
  this topic.  Searching the web for ``latex figure inclusion'' or
  other similar terms will turn up some useful links.

\item[\pkg{natbib}, \pkg{apacite}, \pkg{historian}] These packages
  provide alternatives for formatting citations and bibliographic
  reference lists.  The document in the \complit{sample} folder
  provides more citation examples, and provides links to other
  information regarding bibliography creation.

\item[\pkg{hyperref}] If you are generating an electronic version of
  your document for which you'd like to have hyperlinks automatically
  connecting cross-references and entries in the \textsl{Table of
    Contents}, you should activate the \pkg{hyperref} package.  The
  \pkg{hyperref} package has lots of configuration options, and you
  should refer to its documentation for helpful information.  The use
  of external hyperlinks within your document is discouraged. If
    you enable this package, be sure to turn color links \emph{off},
  as the \booktitle{Guidelines} prohibit multi-colored text in your
  document.
\end{description}

\section{Keeping Things in Order}

If you have lots of figures or musical example files in your document,
you may want to keep these files in the sub-folder already created for
you.  This helps to keep your thesis folder a little less cluttered.
Then if you have a chart called \complit{pie.eps} stored in the
\complit{figures} folder, you just need to include the folder name
when issuing the \verb|\includegraphics|
command\index{includegraphics@\verb+\includegraphics+}, e.g.,
\begin{ltxexample}
\includegraphics{figures/pie.eps}
\end{ltxexample}
You can create any number of folders and sub-folders to help keep
your files organized.

\section{More Examples}
The files in the \complit{thesis-template} directory are only a
bare-bones template to help you get started on your own manuscript.
You will find a more complete example of a thesis manuscript in the
\complit{sample} directory.\index{examples}
The \LaTeX{} source
files in this directory contain explanatory comments and many more
examples of useful code.  The file \complit{thesis.pdf} in the
\complit{sample} directory is the result of processing the source
files, so you can easily compare the source files to the output to see
how everything works.  You'll find some simple examples of equations,
figures, tables, and bibliographic citations to help you create your
own document.  You can find much more help from the web should you
need more sophisticated examples.

\section{Bugs, Corrections, Improvements}

Should you discover what you think is a bug in the way that \fsuth{}
formats your document, you may e-mail me at \texttt{bwhissel@fsu.edu}.
It would be helpful to send the portion of your document which you
believe is misbehaving.  Likewise, if you think that the appearance
of theses or dissertations may be improved in some way, or if you have
some macro definitions that you think may be generally useful and
could be added to \fsuth, I am happy to hear your ideas.

Also, if you think that any of this documentation is misleading or
unclear, \emph{please} let me know.  I wish to make this
\booktitle{User's Guide} and the \fsuth{} class as helpful as
possible.

Please note that I cannot help you to learn features of \LaTeX: there
are many resources and tutorials that are freely available, and I am
unable to support individual requests for help with anything that does
not pertain directly to the \fsuth{} class.

\bigskip
\noindent
Best wishes, and good luck!


\clearpage

\addcontentsline{toc}{section}{Index}
\printindex


\end{document}
