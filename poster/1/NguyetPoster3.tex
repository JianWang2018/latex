%%%%%%%%%%%%%%%%%%%%%%%%%%%%%%%%%%%%%%
% LaTeX poster template
% Created by Nathaniel Johnston
% August 2009
% http://www.nathanieljohnston.com/2009/08/latex-poster-template/
%%%%%%%%%%%%%%%%%%%%%%%%%%%%%%%%%%%%%%

\documentclass[final]{beamer}
%\usepackage[scale=1.25]{beamerposter}
\usepackage[scale=0.8]{beamerposter}
\usepackage{graphicx}			% allows us to import images
\usepackage{multicol}
%\newtheorem{theorem}[theorem]{Theorem}
\newtheorem{acknowledgement}[theorem]{Acknowledgement}
\newtheorem{algorithm}[theorem]{Algorithm}
\newtheorem{axiom}[theorem]{Axiom}
\newtheorem{case}[theorem]{Case}
\newtheorem{claim}[theorem]{Claim}
\newtheorem{conclusion}[theorem]{Conclusion}
\newtheorem{condition}[theorem]{Condition}
\newtheorem{conjecture}[theorem]{Conjecture}
%\newtheorem{corollary}[theorem]{Corollary}
\newtheorem{criterion}[theorem]{Criterion}
%\newtheorem{definition}[theorem]{Definition}
%\newtheorem{example}[theorem]{Example}
\newtheorem{exercise}[theorem]{Exercise}
%\newtheorem{lemma}[theorem]{Lemma}
\newtheorem{notation}[theorem]{Notation}
%\newtheorem{problem}[theorem]{Problem}
\newtheorem{proposition}[theorem]{Proposition}
\newtheorem{remark}[theorem]{Remark}
%\newtheorem{solution}[theorem]{Solution}

\newtheorem{summary}[theorem]{Summary}
\usepackage{multirow}

%\newenvironment{proof}[1][Proof]{\noindent\textbf{#1.} }{\ \rule{0.5em}{0.5em}}
%-----------------------------------------------------------
% Define the column width and poster size
% To set effective sepwid, onecolwid and twocolwid values, first choose how many columns you want and how much separation you want between columns
% The separation I chose is 0.024 and I want 4 columns
% Then set onecolwid to be (1-(4+1)*0.024)/4 = 0.22
% Set twocolwid to be 2*onecolwid + sepwid = 0.464
%-----------------------------------------------------------
%\setlength{\paperwidth}{80cm}
\newlength{\sepwid}
\newlength{\onecolwid}
\newlength{\twocolwid}
\newlength{\threecolwid}
\setlength{\paperwidth}{36in}
\setlength{\paperheight}{24in}
\setlength{\textwidth}{0.98\paperwidth}
\setlength{\textheight}{0.98\paperheight}
\setlength{\sepwid}{0.005\paperwidth}
\setlength{\onecolwid}{0.29\paperwidth}
\setlength{\twocolwid}{0.610\paperwidth}
\setlength{\threecolwid}{0.815\paperwidth}
\setlength{\topmargin}{-0.5in}
%\setlength{\paperwidth}{80cm}
\usetheme{confposter}
\usepackage{exscale}
\newcommand{\R}{\mathbb R}
%-----------------------------------------------------------
% The next part fixes a problem with figure numbering. Thanks Nishan!
% When including a figure in your poster, be sure that the commands are typed in the following order:
% \begin{figure}
% \includegraphics[...]{...}
% \caption{...}
% \end{figure}
% That is, put the \caption after the \includegraphics
%-----------------------------------------------------------

\usecaptiontemplate{
\small
\structure{\insertcaptionname~\insertcaptionnumber:}
\insertcaption}

%-----------------------------------------------------------
% Define colours (see beamerthemeconfposter.sty to change these colour definitions)
%-----------------------------------------------------------
\definecolor{lightorange}{rgb}{0.9,0.4,0}
%\definecolor{lightestorange}{rgb}{1,0.8,0.5}
\setbeamercolor{block title}{fg=lightorange,bg=white}
\setbeamercolor{block body}{fg=black,bg=white}
\setbeamercolor{block alerted title}{fg=white,bg=dblue!70}
\setbeamercolor{block alerted body}{fg=black,bg=dblue!10}

%-----------------------------------------------------------
% Redefine the itemize envinronment
%-----------------------------------------------------------


\makeatletter
\renewcommand{\itemize}[1][]{%
  \beamer@ifempty{#1}{}{\def\beamer@defaultospec{#1}}%
  \ifnum \@itemdepth >2\relax\@toodeep\else
    \advance\@itemdepth\@ne
    \beamer@computepref\@itemdepth% sets \beameritemnestingprefix
    \usebeamerfont{itemize/enumerate \beameritemnestingprefix body}%
    \usebeamercolor[fg]{itemize/enumerate \beameritemnestingprefix body}%
    \usebeamertemplate{itemize/enumerate \beameritemnestingprefix body begin}%
    \list
      {\usebeamertemplate{itemize \beameritemnestingprefix item}}
      {\def\makelabel##1{%
          {%
            \hss\llap{{%
                \usebeamerfont*{itemize \beameritemnestingprefix item}%
                \usebeamercolor[fg]{itemize \beameritemnestingprefix item}##1}}%
          }%
        }%
      }
  \fi%
  \beamer@cramped%
  \justifying% NEW
  %\raggedright% ORIGINAL
  \beamer@firstlineitemizeunskip%
}
\makeatother


%-----------------------------------------------------------
% Name and authors of poster/paper/research
%-----------------------------------------------------------

\title{Hidden Markov Model for Financial Economics}
\author{Nguyet Nguyen, Major Prof: Giray \"{O}kten}
\institute{Financial Mathematics, Department of Mathematics, Florida State University}
%\includegraphics[height=9.0em]{FSU}
%-----------------------------------------------------------
% Start the poster itself
%-----------------------------------------------------------

\begin{document}
\begin{frame}[t]
  \begin{columns}[t]												% the [t] option aligns the column's content at the top
    \begin{column}{\sepwid}\end{column}			% empty spacer column
    \begin{column}{\onecolwid}
      \begin{block}{Introduction}
        A Hidden Markov model (HMM) is a stochastic signal model which has three assumptions:
\begin{enumerate}
\item The observation at time $t$, $O_t$, was generated by some process whose state, $S_t$, is \textbf{hidden}.
\vspace{.5cm}
\item The hidden process satisfies the first-order Markov property: given $S_{t-1}$, $S_t$ is independent of $S_i$ for any $i<t-1$.
\vspace{.5cm}
\item The hidden state variable is discrete.
\vspace{.5cm}
\end{enumerate}
        %\vskip1ex
      \end{block}

      \vskip2ex

      \begin{block}{ Elements of HMM}
     % \begin{itemize}
%Let N be a positive integer, and $\omega=\{\textbf{x}_1,\textbf{x}_2,..,\textbf{x}_N\}$ be a sequence of real numbers.
% \begin{block}{Star-discrepancy}
% The star-discrepancy of $\omega$ is defined by
%$$ D^*(\omega)=\sup_{J\in {\cal J}^*}\left\vert\frac{1}{N}\sum_{i=1}^Nc_J(x_i)-\lambda(J)\right\vert,$$
%where ${\cal J}^*$ is the family of all subintervals of $I^s$ of the form $\prod_{i=1}^s[0,u_i)$, $c_J$ is the characteristic function of $J$, and $\lambda(J)$ is the volume of $J$ with respect to the Lebesque measure.
%\end{block}
%\begin{itemize}
%\item For an arbitrary subset $E$ of $\mathbb{R}$, we define
%\begin{equation}
%A(B;\omega)=\sum_{n=1}^Nc_E(x_n),
%\end{equation}
%where $c_E$ is the characteristic function of $E$.
%\item
 %     \begin{definition}
%  Let ${\cal J}$ be a nonempty family of Lebesgue-measurable subsets of $I^s$, and $\lambda$ a Lebesgue measure.
 %The number
 % \begin{equation}
%D_N=D_N({\cal J},\omega)=\sup_{J\in {\cal J}} \left | \frac{A(J;\omega_N)}{N}-\lambda(J) \right |
%  \end{equation}
 %   is called the discrepancy of $\omega_N$.
 %\end{definition}
%\begin{definition}
%\begin{block}{F-discrepancy}

\begin{enumerate}
 \item Observation data, $O=(O_t)$, $t=1,..,T$
 \vspace{.3cm}
 \item Hidden states, $S=(S_i) ,i=1,2,...,N$
 \vspace{.3cm}
 \item Hidden state sequence: $Q=(q_t)$, $t=1,...,T$
 \vspace{.3cm}
 \item Transition matrix $A$
$$a_{ij}=P(q_t=S_j|q_{t-1}=S_i),~i,j=1,2,...,N$$
%\vspace{.5cm}
\item Observation symbols per state, $V=(v_k), k=1,2,...,M$
\vspace{.5cm}
\item The observation probability
$$B: b_i(k) = P(O_t=v_k|q_t = S_i), i=1,2,...,N; k=1,2,...,M$$
\item Initial probabilities, vector $p$, of being in state $S_i$ at $t=1$ $$  p_i=P(q_1=S_i),~i=1, 2,...,N$$
 \end{enumerate}
 %\includegraphics[width=.83in]{p1-crop.pdf}


\end{block}

   \begin{block}{Three problems and corresponding solutions for HMMs}

\begin{enumerate}
\item Given $(O,\lambda)$, compute the probability of observations, $P(O|\lambda)$\\
%\vspace{.5cm}
\alert{\textbf{Forward, backward algorithm}}
%\vspace{.5cm}
\item Given $(O,\lambda)$, simulate the most likely hidden states, $Q$\\
%\vspace{.5cm}
\alert{ \textbf{ Viterbi algorithm}}
%\vspace{.5cm}
 \item Given $O$, calibrate HMM parameters, $\lambda$\\
% \vspace{.5cm}
\alert{ \textbf{ Baum-Welch algorithm}}
\end{enumerate}
    \end{block}

\begin{block}{Forward algorithm}

\begin{enumerate}
\item Initialization, $\alpha_1(i) = p_i b_i(O_1)$  for $i=1,..., N$
\vspace{.5cm}
\item For $t = 2, 3, . . . , T$, for $j=1,...,N$
 $$\alpha_t(j) = \displaystyle{\left[\sum_{i=1}^N \alpha_{t-1} (i) a_{ij}\right]} b_j (O_t), $$
\item
$P (O|\lambda) =
\sum_{i=1}^N\alpha_T(i)$
\end{enumerate}
\end{block}


 %   \end{column}

   % \begin{column}{\sepwid}\end{column}			% empty spacer column

   % \begin{column}{\onecolwid}

    \end{column}
    \begin{column}{\sepwid}\end{column}			% empty spacer column
    \begin{column}{\onecolwid}
    \begin{block}{Hidden Markov Model}
    \begin{figure}[ht]
\begin{center}
\includegraphics[width=8in]{p11-crop.pdf}
%\includegraphics[width=2in]{BetaAWQMC-crop.pdf}
%\caption{1000 ${\cal B}(.5,.3)$ variate generated by AR-MC (AW)}
\end{center}
\end{figure}
 \end{block}

\begin{block}{Some Applications of HMMs}
 \begin{figure}[ht]
\begin{center}
    \includegraphics[width=2.8in]{speech}
    \includegraphics[width=2.8in]{DNA2}
    \includegraphics[width=2.8in]{fin4}
\end{center}
\caption{1. Speech recognition~~2. Bioinformatics~~3. Finance}
\end{figure}

 \end{block}

 %      \begin{figure}[ht]
% 		\begin{center}
%		\includegraphics[width=5in]{Gamma34MC-crop.pdf}
%		\includegraphics[width=5in]{Gamma34QMC-crop.pdf}
%		\caption{1000 Gamma(3,4) variate generated by Reject-MC (left) and Reject-QMC (right) Cheng-Feast}
%		\end{center}
%		\end{figure}
%    \end{block}
%    \end{column}

%    \begin{column}{\sepwid}\end{column}			
 %   \begin{column}{\onecolwid}
 % \begin{figure}[ht]
%\begin{center}
%\includegraphics[width=2in]{GammaMC-crop.pdf}
%\includegraphics[width=2in]{GammaQMC-crop.pdf}
%\caption{1000 Gamma(.3,3) variate generated by Reject-MC (left) and Reject-QMC (right) Johnk}
%\end{center}
%\end{figure}
\begin{block}{Forecast economics regimes}
%\begin{figure}[ht]
%\begin{center}
%\includegraphics[width=10in]{BetaBBMC-crop.pdf}
%\includegraphics[width=5in]{BetaBBQMC-crop.pdf}
%\caption{1000 ${\cal B}(1.5,1.3)$ variate generated by AR-MC (BB*)}
%\end{center}
%\end{figure}
%\end{block}
%\end{block}
%\end{column}
%\begin{column}{\sepwid}\end{column}			
%    \begin{column}{\onecolwid}
 %   \begin{block}{${\cal B}(1.5,1.3)$-QMC}
%\begin{figure}[ht]
%\begin{center}
%\includegraphics[width=5in]{BetaBBMC-crop.pdf}
%\includegraphics[width=10in]{BetaBBQMC-crop.pdf}
%\caption{1000 ${\cal B}(1.5,1.3)$ variate generated by AR-QMC (BB*)}
%\end{center}
%\end{figure}
Economics indicators
\begin{enumerate}
%\item Equity turbulence
%\item Currency turbulence
%\item Inflation (CPI)
\begin{multicols}{2}
\item Credit Index
\item Yield Curve
\item Commodity
\item Dow Jones Industrial Average
\end{multicols}
%\item Stocks\textbf{Stocks}
%\begin{itemize}
% \item $S\&P 500$, a stock market index based on the market capitalizations of 500 large companies having common stock listed on the NYSE or NASDAQ: daily and monthly prices\\
%\item $SPY$: bid prices in seconds on January 7, 2011
%\end{itemize}
\end{enumerate}
\begin{figure}[ht]
\begin{center}
  \includegraphics[width=4.8in]{DJBear1-crop.pdf}
   %\includegraphics[width=5.5in]{CPI-Bear.pdf}
   \includegraphics[width=4.8in]{NdataBear-crop.pdf}
  % \includegraphics[width=2.1in]{CPI_STATES(HM1).jpg}
\caption{HMMs for Bear Market Predictions}\label{Par}
\end{center}
\end{figure}
\end{block}
\end{column}
\begin{column}{\sepwid}\end{column}			

    \begin{column}{\onecolwid}
\begin{block}{Training and Predicting Process}
 \begin{enumerate}
 \item Use HMM for single and multiple observation data with normal distributions.

 \item Calibrate Markov-switching model parameters using Baum-Welch algorithm

 \item Use the obtained parameters to predict stock prices for the next trading period.
 \end{enumerate}
\end{block}
   \begin{block}{HMMs for Stock Price Predictions}
\begin{figure}[ht]
\begin{center}
    \includegraphics[width=4.5in]{SP500Close-crop.pdf}
    \includegraphics[width=4.5in]{SP500_00703-crop.pdf}
   %\includegraphics[width=2.5in]{GNP_1947_I-2012-II.jpg}
\caption{Forecast $S\& P 500 $ close prices using single observation}
\end{center}
\end{figure}

\end{block}
%\begin{figure}[ht]
%\begin{center}
%\includegraphics[width=in]{BetaAWMC-crop.pdf}
%\includegraphics[width=10in]{BetaAWQMC-crop.pdf}
%\caption{1000 ${\cal B}(.5,.3)$ variate generated by AR-QMC (AW)}
%\end{center}
%\end{figure}

          \begin{block}{HMMs for Stock Tradings}
\begin{table}[h]\label{b1}
 \centering
 \begin{tabular}{c|c|c|c}
  \hline
  Symbol & Initial Investment ($\$$)& Earning ($\$$)&  Earning $\%$\\
 \hline
  \hline
  SPY  &9,000.00 &2050.66 &22.79\\
  \hline
  GOOG &30,000.00& 29,036.4& 96.79\\
  \hline
  FORD& 250.00 & 10.10& 4.04\\
  \hline
   AAPL& 950.00 & 19.06& 2.01\\
  \hline
   GE & 1,700.00& 490.00&28.82\\
  \hline
  \hline
  TOTAL&41,900.00&31,606.22&75.43\\
  \hline
\end{tabular}
\caption{One year daily stock trading portfolio from December 2012 to December 2013}
\label{ziggurat}
\end{table}
          \end{block}
          %\begin{block}{Future work}
         % \end{block}
          \begin{block}{References}	
            \small{\begin{thebibliography}{99}
		   \bibitem{book1} Zoubin Ghahramani. An Introduction to Hidden Markov Models and Bayesian Networks. International Journal of Pattern Recognition and Artifical Interlligence, 15(1): 9-42, 2001.
\bibitem{} Lawrence Rabiner. A Tutorial on Hidden Markov Models and Seclected Applications in Speech Recognition, IEEE Vol 77-2, 1989.
\bibitem{} Przemyslaw Dymarski. Hidden Markov Models, Theory and Applications, InTech 2011.
\bibitem{} Xiaolin Li, Training Hidden Markov Models with Multiple Observations – A Combinatorial Method. IEEE Transactions on PAMI, vol. 22(4): 371-377, 2000.
\bibitem{} Mark Kritzman, Sébastien Page, and David Turkington. Regime Shifts: Implications for
Dynamic Strategies. Financial Analysts Journal, Vol.68(3), 2012 CFA Institute.
		        \end{thebibliography}}
		      \end{block}
    \end{column}
  \begin{column}{\sepwid}\end{column}			% empty spacer column
 \end{columns}
\end{frame}
\end{document}
