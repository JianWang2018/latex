\chapter{Two Classes Arbitrage Prediction Results}
In chapter 6.4, we list the result of  ask-low arbitrage opportunity for stock AMZN based on Logistic regression(Lasso penalty), Logistic regression(Ridge penalty),support vector machine,decision tree, Adaboost, and Random forest.  In this appendix, we show the results of the other four stocks including Apple(AAPL),Google(GOOG),Intel(INTC), and Microsoft(MSFT). \\

\begin{table}[htp!]
	\caption{AAPL ask-low arbitrage opportunity prediction(5 seconds)}
	\label{ask_low_prediction}
	\begin{center}
		\scalebox{0.8}{
		\begin{tabular}{|c|c|c|c|c|c|c|}
			\hline
			 Model& Training & Training  & Test  & Test & Test & Test \\[5pt]
			 & time(s) & F1 score & time(s) & Recall & Precision & F1 score \\[5pt]
			 \hline
			 Logistic regression(Lasso penalty)& 389.8 & 9.3 \% & 0.002 & 3.0\% & 41.7\%& 5.5\%\\[5pt]
			 Logistic regression(Ridge penalty)& 5.7 & 9.3 \% & 0.01 &  3.0\% & 41.7\%& 5.5\%\\[5pt]
			 SVM(Poly 2 kernal,5000 estimator)& 105.7 & 68.7 \% & 8.6 &  36.1\% & 96.1\%& 52.5\%\\[5pt]
			 Decision Tree(no pruning)& 4.3 & 50.6 \% & 0.003 &  20.4\% & 97.2\%& 33.7\%\\[5pt]
			 Adaboost(number of estimate=100)& 32.2 & 90.1 \% & 0.03 & 71.0 \% & 84.5\% &  77.2\% \\[5pt]
			 Random forest(number of estimate=100)& 45.9 & 99.5 \% & 0.13 & 71.3 \% & 98.4\% &  82.7\% \\[5pt]		 	
	 		\hline 
		\end{tabular}
	}
	\end{center}
\end{table}


\begin{table}[htp!]
	\caption{GOOG ask-low arbitrage opportunity prediction(5 seconds)}
	\label{ask_low_prediction}
	\begin{center}
	\scalebox{0.8}{
		\begin{tabular}{|c|c|c|c|c|c|c|}
			\hline
			 Model& Training & Training  & Test  & Test & Test & Test \\[5pt]
			 & time(s) & F1 score & time(s) & Recall & Precision & F1 score \\[5pt]
			 \hline
			 Logistic regression(Lasso penalty)& 392.9 & 3.8 \% & 0.002 & 4.3\% & 83.3\%& 8.2\%\\[5pt]
			 Logistic regression(Ridge penalty)& 5.6 & 3.8 \% & 0.01 &  4.3\% & 83.3\%& 9.2\%\\[5pt]
			 SVM(Poly 2 kernal,5000 estimator)& 64.2 & 36.8 \% & 3.3 &  20.7\% & 100.0\%& 34.3\%\\[5pt]
			 Decision Tree(no pruning)& 4.6 & 54.6 \% & 0.003 &  50.0\% & 96.7\%& 65.9\%\\[5pt]
			 Adaboost(number of estimate=100)& 40.2 & 96.1 \% & 0.03 & 76.7 \% & 97.8 &  86.0\% \\[5pt]
			 Random forest(number of estimate=100)& 4.4 & 79.5 \% & 0.01 & 77.6 \% & 98.9\% &  87.0\% \\[5pt]		 	
	 		\hline 
		\end{tabular}
}
	\end{center}
\end{table}

\begin{table}[htp!]
	\caption{INTC ask-low arbitrage opportunity prediction(5 seconds)}
	\label{ask_low_prediction}
	\begin{center}
		\scalebox{0.8}{
		\begin{tabular}{|c|c|c|c|c|c|c|}
			\hline
			 Model& Training & Training  & Test  & Test & Test & Test \\[5pt]
			 & time(s) & F1 score & time(s) & Recall & Precision & F1 score \\[5pt]
			 \hline
			 Logistic regression(Lasso penalty)& 337.7 & 6.3 \% & 0.002 & 3.2\% & 1.4\%& 2.0\%\\[5pt]
			 Logistic regression(Ridge penalty)& 12.1 & 62.8 \% & 0.01 &  3.2\% & 15.8\%& 2.1\%\\[5pt]
			 SVM(Poly 2 kernal,5000 estimator)& 42.9 & 74.3 \% & 3.0 &  3.2\% & 10.0\%& 4.8\%\\[5pt]
			 Decision Tree(no pruning)& 2.3 & 91.3 \% & 0.003 &  56.8\% & 69.6\%& 62.5\%\\[5pt]
			 Adaboost(number of estimate=100)& 198.0 & 99.9 \% & 0.20 & 85.2 \% & 83.9 &  84.6\% \\[5pt]
			 Random forest(number of estimate=100)& 12.5 & 99.9 \% & 0.1 & 65.6 \% & 80.4\% &  72.2\% \\[5pt]		 	
	 		\hline 
		\end{tabular}
	}
	\end{center}
\end{table}

\begin{table}[htp!]
	\caption{MSFT ask-low arbitrage opportunity prediction(5 seconds)}
	\label{ask_low_prediction}
	\begin{center}
		\scalebox{0.8}{
		\begin{tabular}{|c|c|c|c|c|c|c|}
			\hline
			 Model& Training & Training  & Test  & Test & Test & Test \\[5pt]
			 & time(s) & F1 score & time(s) & Recall & Precision & F1 score \\[5pt]
			 \hline
			 Logistic regression(Lasso penalty)& 358.6 & 39.5 \% & 0.03 & 15.9\% & 50.0\%& 24.1\%\\[5pt]
			 Logistic regression(Ridge penalty)& 10.2 & 39.5 \% & 0.002 &  15.9\% & 50\%& 24.1\%\\[5pt]
			 SVM(Poly 2 kernal,5000 estimator)& 88.3 & 77.8 \% & 6.3 &  36.4\% & 94.1\%& 52.5\%\\[5pt]
			 Decision Tree(no pruning)& 1.9 & 88.1 \% & 0.004 &  52.2\% & 15.9\%& 24.3\%\\[5pt]
			 Adaboost(number of estimate=100)& 132.3 & 99.9 \% & 0.18 & 59.1 \% & 100.0 &  74.28\% \\[5pt]
			 Random forest(number of estimate=100)& 16.4 & 99.9 \% & 0.1 & 59.1 \% & 100.0\% &  74.28\% \\[5pt]		 	
	 		\hline 
		\end{tabular}
	}
	\end{center}
\end{table}

\chapter{PnL and Cumulative PnL}
Figure \ref{fig:pnl} and figure \ref{fig:cum_pnl} show the profit and loss results for four stocks, the trading cost here is set to 0. Methodology is random forest one vs. rest algorithm

\begin{figure}[hbtp]
  \begin{center}
    \includegraphics[width=6in,height=5in]{figures/pnl_total.png}
  \end{center}
\caption{Profit and Loss,  X-axis represents the predicted arbitrage index and Y-axis is profit or loss for each transaction.The top left panel shows the result of AMZN, the top right panel shows the result of GOOG,the bottom left panel shows the result of INTC, and the bottom right panel shows the result of MSFT. } \label{fig:pnl}
\end{figure}

\begin{figure}[hbtp]
  \begin{center}
    \includegraphics[width=6in,height=5in]{figures/cum_pnl_total.png}
  \end{center}
\caption{Profit and Loss,  X-axis represents the predicted arbitrage index and Y-axis is profit or loss for each transaction.The top left panel shows the result of AMZN, the top right panel shows the result of GOOG,the bottom left panel shows the result of INTC, and the bottom right panel shows the result of MSFT.} \label{fig:cum_pnl}
\end{figure}
				
