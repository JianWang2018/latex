\chapter{Latex examples}

This is an introductory paragraph. summary all the situation and example in latex file, such as how to write algorithm, how to input programming code in latex file,etc.

Each section contains one aspect of how to use latex, for example, we write the example of input code in latex file in the first section.

\section{Example of input code}\label{sec:input_code}
In section \ref{sec:input_code}, we show the example of input code:

\begin{lstlisting}
// Hello.java
import javax.swing.JApplet;
import java.awt.Graphics;

public class Hello extends JApplet {
public void paintComponent(Graphics g) {
g.drawString("Hello, world!", 65, 95);
}    
}
\end{lstlisting}


\section{Example of algorithm}
Example of algorithm is as follows, using the package of algorithm2e:\\

\begin{algorithm}[H]
\caption{My algorithm}\label{euclid}
		    \nl Initialize the observation weights $\omega_i$=$1/N$,$i$=1,2,...,$N$\;
		    \nl \For{m=1 to M}{
		    \nl Fit a classifer $G_m(x)$ to the training data using weights $\omega_i$\;
		    \nl   Compute \;
		      \begin{equation*}		      
		       err_m=\frac{\sum_{i=1}^{N}\omega_i I(y_i \neq G_m(x_i))}{\sum_{i=1}^{N}\omega_i}		       
		      \end{equation*} 
		    \nl Compute $\alpha_m=log((1-err_m)/err_m)$\;
		    \nl Set $\omega_i \gets \omega_i \cdot exp[\alpha_m \cdot I(y_i \neq G_m(x_i))]$, $i =1,2,...,N$ \;
		    }
		    \nl Output $G(x)=sign[\sum_{m=1}^{M}\alpha_m G_m(x)]$ \;
\end{algorithm}

\section{Example of array}
\begin{equation*}
X=\left\{
\begin{aligned}
a+b&=2\\
a-b&=4\\
\end{aligned}
\right.
\end{equation*}
\section{cite reference}

This is my abstract \cite{cont2010stochastic}

\section{example of table}
\begin{table}
	\caption{Table example}
	\label{sonnets}
	\begin{center}
		\begin{tabular}{|r| p{5cm} |}
			\hline
			8 & Music to hear, why hear'st thou music sadly? \\[5pt]
			9 & Is it for fear to wet a widow's eye \\[5pt]
			10 & For shame deny that thou bear'st love to any \\[5pt]
			11 & As fast as thou shalt wane, so fast thou grow'st \\[5pt]
			12 & When I do count the clock that tells the time \\
			\hline 
		\end{tabular}
	\end{center}
\end{table}