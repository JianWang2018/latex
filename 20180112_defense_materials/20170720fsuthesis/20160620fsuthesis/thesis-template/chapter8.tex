\chapter{Conclusion and Future Works}\label{ch:conclusion}
This paper designs a framework for exploring the dynamics of LOBs based on ensemble machine learning methods. With significantly better performance relative to the support vector machine, logistic regression with lasso and ridge penalty, decision tree, and ensemble methods are found to  have good applicability in predicting price spread crossing opportunities of LOBs. The strong outperformance of ensemble methods over other models suggests that the existing financial markets' arbitrage strategies can be strengthened by using ensemble methods. Our models are trained and tested with substantial data samples including the total transaction histories for one trading day of five information technology stocks. The amount of data for these five stocks ranges from two hundred thousand to six hundred thousand. To deal with the imbalanced data problem, we use precision, recall, and F1 scores to measure the performance of different models. In order to deal with the multi-category classification problem, we use one vs. one and one vs. rest methods. The confusion matrices of ensemble methods based on our problem show that the framework of our research also works well for the multi-classification problem.
The experimental results with real-time stock data show that the proposed frameworks based on ensemble methods are effective. 
In addition, a simple trading strategy that uses the predicting results of our models can make a significant profit with low risks, indicating that our model is beneficial for the actual transaction. 

The framework of our research can be extended in several ways. One possible research direction is to add more features that related to changes in stock prices. For example, \cite{wahal2013style} show that some variables, such as book-to-market ratio and size are helpful for predicting stock prices. Hence, there are a lot of potential alternative features for future research and how to avoid over-fitting becomes an interesting problem. Some past related issues can be found in \cite{fan2008sure}, \cite{fan2010sure}, and \cite{buhlmann2013correlated}.

Another possible future research orientation is to apply our models to asset allocation and  portfolio management. In our paper, the PnL of a trading strategy based on our model in five seconds is profitable. We can test the PnL based on different holding periods such as one second, ten seconds, and one minute. If our models show a profitable performance for all the holding periods, we can design a trading strategy that uses our model in different trading horizons, which may diversify risk by matching the duration of different assets, as well as obtaining considerable profit.  

We can also introduce some of the latest machine learning tools  in the future. For example, deep learning and reinforcement learning will be excellent candidates. \cite{sirignano2016deep} gives some demonstration of how to use deep learning for modeling the spatial distribution of LOBs.  Finally, it is feasible to apply the similar framework in order to model the trends of other financial markets, such as options, futures, and currency exchanges. We can compare the profit results of our framework based on different financial markets in order to test the model's robustness.
 
