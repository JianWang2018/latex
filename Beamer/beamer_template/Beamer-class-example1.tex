% This text is proprietary.
% It's a part of presentation made by myself.
% It may not used commercial.
% The noncommercial use such as private and study is free
% Sep. 2005 
% Author: Sascha Frank 
% University Freiburg 
% www.informatik.uni-freiburg.de/~frank/



\pdfminorversion=5
\pdfobjcompresslevel=3 
\pdfcompresslevel=9

% Full presentation (with overlays, animated bullet items etc)
\documentclass[xcolor={x11names,svgnames,dvipsnames}]{beamer}

% Transparency mode (no overlays)
%\documentclass[xcolor={x11names,svgnames,dvipsnames},trans]{beamer}

% 4-up handout mode
%\documentclass[xcolor={x11names,svgnames,dvipsnames},handout]{beamer}

\usepackage{pgfpages}

\usepackage[british]{babel}
\usepackage{etex}

%% Glossy pretty look for the presentation and transparency (w/o overlays and animations) versions!
\mode<beamer|trans>{
	\useoutertheme[glossy]{wuerzburg}
	\useinnertheme[shadow,outline]{chamfered}
	\usecolortheme{shark}
}
\setbeamertemplate{navigation symbols}{}
\setbeamertemplate{frametitle continuation}[from second][(cont'd)]
\usefonttheme[stillsansseriftext,stillsansserifsmall]{serif}

%% Save up on ink for the 4-up handouts
\mode<handout>{
	\useoutertheme{wuerzburg}
	\useinnertheme[outline]{chamfered}
	\pgfpagesuselayout{4 on 1}[a4paper, landscape, border shrink=10mm]
	\pgfpageslogicalpageoptions{1}{border code=\pgfstroke}
	\pgfpageslogicalpageoptions{2}{border code=\pgfstroke}
	\pgfpageslogicalpageoptions{3}{border code=\pgfstroke}
	\pgfpageslogicalpageoptions{4}{border code=\pgfstroke}
}

\mode<presentation>{\AtBeginSection{%
		\begin{frame}
			\frametitle{Contents}
			\tableofcontents[currentsection]
		\end{frame}}}
		
		\usepackage{microtype}
		\usepackage[utf8]{inputenc}
		\usepackage[T1]{fontenc}
		\usepackage[osfss]{libertine}
		\usepackage[scaled=.77]{beramono}
		\usepackage{cmap}
		
		\usepackage{relsize,tabularx}
		\usepackage[T1,safe]{tipa}
		\usepackage{dtklogos,hologo,textcomp}
		\usepackage{multicol,booktabs}
		\usepackage{listings}
		\lstset{upquote,keepspaces=true,columns=spaceflexible,
			basicstyle=\ttfamily\scriptsize,%
			breaklines=true,breakindent=0pt,xleftmargin=0pt, xrightmargin=6pt,%
			language=[LaTeX]TeX, texcsstyle=*\bfseries\color{Maroon}, commentstyle=\sffamily\itshape\smaller\color{SeaGreen4},
			emphstyle=\bfseries\color{RoyalBlue3},escapechar={:},
			emphstyle={[2]{\bfseries\color{Sienna2}}},
			postbreak=\mbox{{\smaller\color{gray}$\hookrightarrow$}}
		}
		
		\usepackage{tikz}
		\usetikzlibrary{shapes,arrows,positioning,matrix,chains,backgrounds,fit}
		
		\usepackage{multicol}
		\usepackage[version=3]{mhchem}
		\usepackage{chemfig}
		\usepackage{linguex,qtree}
		\let\fg\lingfg
		\usepackage{texshade}
		\usepackage[detect-all]{siunitx}
		\usepackage[siunitx]{circuitikz}
		\usepackage{bytefield}
		\usepackage{auto-pst-pdf}
		\usepackage{pstricks,pst-barcode}
		\usepackage{pgfplots}
		\usepackage{pgfgantt}
		\usepackage[skaknew]{chessboard,skak}
		\usepackage{cwpuzzle}
		\usepackage{gchords,guitar}
		\usepackage{spreadtab}
		\usepackage{ccicons}


\setlength\fboxsep{0pt}
\SetTracking{encoding=*}{-39}

\author[\textsc{Jian} Wang]{\textsc{Jian} Wang\\[1ex]%
{\small\url{wangjian790@gmail.com}\\[-.5ex]\url{}}\\
{\small{Financial math Ph.D Candidate}}\\
{\small{Florida State University}}\\
[0.8ex]\copyright\copyright\copyright\copyright} %\ccbyncsa}

\title{High frequency data trading}
%\titlegraphic{\ccbyncsa}
\date[\textsc{HFC} 2015]{High Frequency data Conference data Conference 2015 }%

% Include QR code of slides URL for presentation-mode only -- so that audience
%  can snap it and download it on the spot
\mode<beamer>{\titlegraphic{\begin{pspicture}\psbarcode[scalex=.75,scaley=.75]{http://liantze.penguinattack.org/latextypesetting.html\#mosc11-slides}{eclevel=L}{qrcode}\end{pspicture}}}

\hypersetup{%
pdfauthor={Jian Wang}, %% the "author" field from above includes garbage code...
pdfkeywords={HFC,Machine Learning,general}
}
\begin{document}
\begin{frame}
\maketitle
\end{frame}

%\title{Simple Beamer Class}   
%\author{Sascha Frank} 
%\date{\today} 

\frame{\frametitle{Table of contents}\tableofcontents} 


\section{Section no.1} 
\frame{\frametitle{Title} 
Each frame should have a title.
}
\subsection{Subsection no.1.1  }
\frame{ 
Without title somethink is missing. 
}
%
\newcommand<>{\hover}[1] {\uncover#2 {
	\begin{tikzpicture}[remember picture,overlay,fill opacity=1]
    \draw [fill opacity=1] (current page.south west)
	rectangle (current page.north east);
	\node at (current page.center) {#1};
	\end{tikzpicture}}
	}

\begin{frame}
   \frametitle{A frame}
      Some text
      \hover<2>{
      \begin{minipage}{0.8\linewidth}
        \begin{block}{A block hovering above the slide}
        I am visible on slide two.
        \end{block}
      \end{minipage}
      }
\end{frame}
\section{Section no. 2} 
\subsection{Lists I}
\frame{\frametitle{unnumbered lists}
\begin{itemize}
\item Introduction to  \LaTeX  
\item Course 2 
\item Termpapers and presentations with \LaTeX 
\item Beamer class
\end{itemize} 
}

\frame{\frametitle{lists with pause}
\begin{itemize}
\item Introduction to  \LaTeX \pause 
\item Course 2 \pause 
\item Termpapers and presentations with \LaTeX \pause 
\item Beamer class
\end{itemize} 
}

\subsection{Lists II}
\frame{\frametitle{numbered lists}
\begin{enumerate}
\item Introduction to  \LaTeX  
\item Course 2 
\item Termpapers and presentations with \LaTeX 
\item Beamer class
\end{enumerate}
}
\frame{\frametitle{numbered lists with pause}
\begin{enumerate}
\item Introduction to  \LaTeX \pause 
\item Course 2 \pause 
\item Termpapers and presentations with \LaTeX \pause 
\item Beamer class
\end{enumerate}
}

\section{Section no.3} 
\subsection{Tables}
\frame{\frametitle{Tables}
\begin{tabular}{|c|c|c|}
\hline
\textbf{Date} & \textbf{Instructor} & \textbf{Title} \\
\hline
WS 04/05 & Sascha Frank & First steps with  \LaTeX  \\
\hline
SS 05 & Sascha Frank & \LaTeX \ Course serial \\
\hline
\end{tabular}}


\frame{\frametitle{Tables with pause}
\begin{tabular}{c c c}
A & B & C \\ 
\pause 
1 & 2 & 3 \\  
\pause 
A & B & C \\ 
\end{tabular} }


\section{Section no. 4}
\subsection{blocs}
\frame{\frametitle{blocs}

\begin{block}{title of the bloc}
bloc text
\end{block}

\begin{exampleblock}{title of the bloc}
bloc text
\end{exampleblock}


\begin{alertblock}{title of the bloc}
bloc text
\end{alertblock}
}
\end{document}

